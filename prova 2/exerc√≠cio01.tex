\section{Produto tensorial sobre espaços lineares}
\subsection{Quocientes de grupos}
\begin{definition}{Subgrupo normal}{subgrupo_normal}
    Seja \(G\) um grupo. Um subgrupo \(N\) de \(G\) é \emph{normal}, denotado por \(N \normal G\), se for invariante por conjugações, isto é, se \(n \in N\) então \(g^{-1} n g \in N\) para todo \(g \in G\).
\end{definition}

\begin{proposition}{Todo subgrupo de um grupo abeliano é normal}{subgrupo_abeliano}
    Seja \(G\) um grupo abeliano e \(N\) um subgrupo. Então \(N \normal G\).
\end{proposition}
\begin{proof}
    Seja \(n \in N\), então \(g^{-1}ng = ng^{-1}g = n \in N\) para todo \(g \in G\), isto é, \(N \normal G\).
\end{proof}

\begin{proposition}{Relação de equivalência para subgrupos normais}{normal_equivalência}
    Seja \(G\) um grupo e \(N \normal G\) um subgrupo normal. Então
    \begin{equation*}
        x \sim_N y \iff x^{-1}y \in N
    \end{equation*}
    define uma relação de equivalência em \(G\).
\end{proposition}
\begin{proof}
    Como \(N\) é um subgrupo, então \(e \in N\), logo a relação é reflexiva pois \(x^{-1}x = e \in N\), isto é, \(x \sim_N x\). Temos
    \begin{equation*}
        x \sim_N y \iff x^{-1} y \in N \iff (x^{-1}y)^{-1} \in N \iff y^{-1}x \in N \iff y \sim_N x,
    \end{equation*}
    portanto a relação é simétrica. Supondo que \(x \sim_N y\) e \(y \sim_N z\) temos
    \begin{equation*}
        x^{-1}y \in N \land y^{-1}z \in N \implies x^{-1}y y^{-1} z \in N \implies x^{-1} z \in N,
    \end{equation*}
    portanto \(x \sim_N z\), isto é, a relação é transitiva.
\end{proof}

\begin{lemma}{Classe de equivalência em relação a um subgrupo normal}{classe_de_equivalência}
    Seja \(G\) um grupo e \(N\normal G\) um subgrupo normal. Para um dado \(g \in G\) denotemos
    \begin{equation*}
        gN = \setc{h \in G}{\exists n \in N : h = gn}\quad\text{e}\quad
        Ng = \setc{h \in G}{\exists n \in N : h = ng},
    \end{equation*}
    então \(gN = [g]_N = Ng\).
\end{lemma}
\begin{proof}
    Temos
    \begin{equation*}
        h \in gN \iff \exists n \in N : g^{-1}h = n \in N \iff h \sim_N g \iff h \in [g]_N
    \end{equation*}
    e
    \begin{equation*}
        h \in Ng \iff \exists n \in N : h = n g \iff \exists n \in N : g^{-1} h = g^{-1} n g \iff g^{-1} h \in N \iff h \in [g]_N
    \end{equation*}
    portanto \(gN = [g]_N = Ng\).
\end{proof}

\begin{proposition}{Grupo quociente}{grupo_quociente}
    Seja \(G\) um grupo e \(N \normal G\) um subgrupo normal. Com o produto definido por
    \begin{align*}
        \cdot : (G/N) \times (G/N) &\to G/N\\
                     ([g]_N,[h]_N) &\mapsto [gh]_N,
    \end{align*}
    o quociente \(G / N = \setc{[g]_N}{g \in G}\) é um grupo. Ainda, se \(G\) é abeliano, então \(G/N\) também o é.
\end{proposition}
\begin{proof}
    Sejam \(g, h \in G\), \(\tilde{g} \in [g]_N\) e \(\tilde{h} \in [h]_N\), então pelo \cref{lem:classe_de_equivalência} temos \(\tilde{g} \in gN\) e \(\tilde{h} \in Nh\), isto é, existem \(n_g,n_h \in N\) tais que \(\tilde{g} = gn_g\) e \(\tilde{h} = n_hh\). Com isso temos
    \begin{equation*}
        (gh)^{-1} \tilde{g}\tilde{h} = h^{-1} g^{-1} \tilde{g} \tilde{h} = h^{-1} n_g n_h h \in N,
    \end{equation*}
    isto é, \(\tilde{g}\tilde{h} \in [gh]_N\). Mostramos assim que o produto é bem definido, por independer da escolha de representante das classes de equivalência.

    Sejam \(g_1,g_2,g_3 \in G\), então
    \begin{equation*}
        [g_1]_N \cdot \left([g_2]_N \cdot [g_3]_N\right) = [g_1]_N \cdot [g_2 g_3]_N = [g_1(g_2 g_3)]_N = [g_1 g_2 g_3]_N
    \end{equation*}
    e
    \begin{equation*}
        \left([g_1]_N \cdot [g_2]_N\right) \cdot [g_3]_N = [g_1 g_2]_N \cdot [g_3]_N = [(g_1 g_2)g_3]= [g_1 g_2 g_3]_N
    \end{equation*}
    portanto o produto em \(G/N\) é associativo. Notemos que \([e]_N = N\) é o elemento neutro deste grupo pois
    \begin{equation*}
        [g]_N [e]_N = [ge]_N = [g]_N = [eg]_N = [e]_N [g]_N
    \end{equation*}
    para todo \(g \in G\). Assim, se \(g \in G\), então o elemento inverso de \([g]_N\) é \([g^{-1}]_N\) já que
    \begin{equation*}
        [g]_N [g^{-1}]_N = [gg^{-1}]_N = [e]_N = [g^{-1}g]_N = [g^{-1}]_N [g]_N,
    \end{equation*}
    portanto \(G/N\) é um grupo. Se \(G\) é abeliano, temos
    \begin{equation*}
        [g]_N [h]_N = [gh]_N = [hg]_N = [h]_N [g]_N
    \end{equation*}
    para todos \(g, h \in G\), isto é, \(G/N\) é abeliano.
\end{proof}

\begin{proposition}{Quociente de módulos}{espaço_quociente}
    Seja \(M\) um módulo sobre um anel \(K\) e seja \(N\) um submódulo de \(M\). Com a adição e o produto por escalares definidos por
    \begin{align*}
        + : M/N \times M/N &\to M/N&
        \cdot : K \times M/N &\to M/N\\
        ([v]_N,[u]_N) &\mapsto [v + u]_N&
        (\alpha,[v]_N)&\mapsto [\alpha v]_N
    \end{align*}
    o quociente \(M/N\) é um módulo sobre \(K\).
\end{proposition}
\begin{proof}
     Sabemos que \((M/N, +)\) é um grupo abeliano aditivo pela \cref{prop:grupo_quociente}. Seja \(v \in M\), então
    \begin{equation*}
        \tilde{v} \in [v]_N \iff \tilde{v} - v \in N \iff \forall \alpha \in K: \alpha\tilde{v} - \alpha v \in N \iff \alpha \tilde{v} \in [\alpha v]_N,
    \end{equation*}
    portanto o produto por escalares está bem definido. Sejam \(\alpha, \beta \in K\) e \(v \in M\), então
    \begin{equation*}
        \alpha \cdot \left(\beta \cdot [v]_N\right) = \alpha \cdot [\beta v]_N = [\alpha (\beta v)]_N = [\alpha \beta v]_N = [(\alpha \beta) v]_N = (\alpha \beta)\cdot [v]_N,
    \end{equation*}
    portanto o produto por escalares é compatível com a multiplicação no corpo. Se \(1 \in K\) é a identidade multiplicativa de \(K\), então \([v]_N = [1v]_N = 1 \cdot [v]_N\), portanto é também a identidade do produto por escalares. Sejam \(u,v \in M\) e \(\alpha, \beta \in K\), então
    \begin{equation*}
        \alpha \cdot \left([u]_N + [v]_N\right) = \alpha \cdot [u + v]_N = [\alpha (u + v)]_N = [\alpha u + \alpha v]_N = [\alpha u]_N + [\alpha v]_N = \alpha\cdot [u]_N + \alpha \cdot [v]_N
    \end{equation*}
    e
    \begin{equation*}
        (\alpha + \beta)\cdot [v]_N = [(\alpha + \beta)v]_N = [\alpha v + \beta v]_N = [\alpha v]_N + [\beta v]_N = \alpha\cdot[v]_N + \beta\cdot[v]_N,
    \end{equation*}
    isto é, o produto por escalares é distributivo em relação à soma no corpo e à soma em \(M/N\). Concluímos portanto que \((M/N, +, \cdot)\) é um módulo sobre \(K\).
\end{proof}
\begin{remark}
    Como todo corpo é um anel, e como um módulo sobre um corpo é um espaço linear, concluímos que o quociente de um espaço linear sobre um corpo \(\mathbb{K}\) por um subespaço linear é um espaço linear sobre \(\mathbb{K}\).
\end{remark}
\subsection{Grupo abeliano livremente gerado por um conjunto}
\begin{lemma}{Grupo abeliano é naturalmente um módulo sobre \(\mathbb{Z}\)}{módulo_inteiros}
    Seja \(G\) um grupo abeliano cujo produto denotaremos por \(+\). Definindo o produto por escalares por
    \begin{align*}
        \cdot : \mathbb{Z} \times G &\to G\\
                               (n,g)&\mapsto \sum_{i = 1}^n g,
    \end{align*}
    com \(0\cdot g = e\), então \(G\) é um módulo sobre \(\mathbb{Z}\).
\end{lemma}
\begin{proof}
    Da definição temos \(1\cdot g = g\) para todo \(g \in G\). Sejam \(n, m \in \mathbb{Z}\) e \(g, h \in G\), então
    \begin{equation*}
        n\cdot(m \cdot g) = \sum_{i = 1}^n (m \cdot g) = \sum_{i = 1}^n \sum_{j = 1}^m g = \sum_{k = 1}^{nm} g = (nm)\cdot g,
    \end{equation*}
    \begin{equation*}
        n\cdot(g + h) = \sum_{i = 1}^n (g + h) = \sum_{i = 1}^n g + \sum_{i = 1}^{n} h = n\cdot g + n \cdot h,
    \end{equation*}
    e
    \begin{equation*}
        (n + m)\cdot g = \sum_{i = 1}^{n + m} g = \sum_{i = 1}^n g + \sum_{i = n+1}^{n + m} g = n \cdot g + \sum_{i = 1}^m g = n\cdot g + m \cdot g.
    \end{equation*}
    Como \(G\) já é um grupo abeliano, verificamos que é um módulo sobre \(\mathbb{Z}\).
\end{proof}
\begin{remark}
    É claro que todo módulo sobre \(\mathbb{Z}\) é um grupo abeliano em relação a sua soma, portanto podemos sempre intercambiar grupos abelianos e módulos sobre \(\mathbb{Z}\) conforme conveniente.
\end{remark}

\begin{definition}{Suporte de uma aplicação}{suporte}
    Seja \(f : X \to G\) uma aplicação de um conjunto não vazio \(X\) em um grupo \(G\) cujo elemento neutro é \(e \in G\). O \emph{suporte de \(f\)} é o conjunto \(\supp(f) = \setc{x \in X}{f(x) \neq e}\). Uma aplicação é dita ser de \emph{suporte finito} se seu suporte for um conjunto finito.
\end{definition}

\begin{proposition}{Grupo abeliano livremente gerado por um conjunto}{grupo_livremente_gerado}
    Seja \(X\) um conjunto não vazio e seja \(\mathfrak{F}(X) = \setc{f \in \mathbb{P}(X \times \mathbb{Z})}{f\text{ é de suporte finito}}.\) Com a soma definida tal que \((f+g)(x) = f(x) + g(x)\), e o produto por escalares definido de tal sorte que \((n\cdot f)(x) = n f(x)\), então \(\left(\mathfrak{F}(X), +, \cdot\right)\) é um módulo livre\footnote{Um módulo é livre se possui uma base de Hamel, um conjunto gerador linearmente independente. Como \(\mathbb{Z}\) não é um anel de divisão, módulos sobre \(\mathbb{Z}\) não têm garantia da existência de uma base, como é o caso em, por exemplo, espaços lineares.} sobre \(\mathbb{Z}\).
\end{proposition}
\begin{proof}
    Sejam \(f, g \in \mathfrak{F}(X)\), então é claro que \(f+g= g+ f\) é uma aplicação de \(X\) em \(\mathbb{Z}\), mas devemos mostrar que \(\supp(f+g)\) é um conjunto finito. Seja \(x \notin \supp(f) \cup \supp(g)\), então \(f(x) = 0\) e \(g(x) = 0\), portanto \(f(x) + g(x) = 0\), isto é, \(x \notin \supp(f + g)\). Isso nos mostra que \(\supp(f+g) \subset \supp(f) \cup \supp(g)\), portanto \(f+g\) é de suporte finito, já que a união de conjuntos finitos é um conjunto finito. A associatividade desta soma é evidente pela associatividade da soma nos inteiros, já que
    \begin{equation*}
        (f_1 + (f_2 + f_3))(x) = f_1(x) + (f_2 + f_3)(x) = f_1(x) + f_2(x) + f_3(x) = (f_1 + f_2)(x) + f_3(x) = ((f_1 + f_2) + f_3)(x)
    \end{equation*}
    para todos \(f_1, f_2, f_3 \in \mathfrak{F}(X)\) e \(x \in X\). Como o suporte da função identicamente nula é o conjunto vazio, \(0 \in \mathfrak{F}(X)\) e então é claramente o elemento neutro desta soma. Assim, o elemento inverso de \(f \in \mathfrak{F}(X)\) é \(-f \in \mathfrak{F}(X)\), mostrando assim que \(\mathfrak{F}(X)\) é um grupo abeliano aditivo. Pelo \cref{lem:módulo_inteiros}, sabemos que \(\mathfrak{F}(X)\) é um módulo sobre \(\mathbb{Z}\).

    % É claro que se \(f \in \mathfrak{F}(X)\), então \(n\cdot f\) tem suporte finito para todo \(n \in \mathbb{Z}\), já que se \(n \neq 0\) o suporte é inalterado e se \(n = 0\) temos \(n\cdot f = 0 \in \mathfrak{F}(X)\). Com a definição dada, é evidente que a identidade multiplicativa em \(\mathbb{Z}\) é a identidade do produto por escalares. Sejam \(n, m \in \mathbb{Z}\), \(f, g \in \mathfrak{F}(X)\), então
    % \begin{equation*}
    %     \left(n\cdot(f+ g)\right)(x) = n (f+g)(x) = nf(x) + ng(x) = (n\cdot f)(x) + (n\cdot g)(x),
    % \end{equation*}
    % \begin{equation*}
    %     \left((n + m)\cdot f\right)(x) = (n + m) f(x) = nf(x) + mf(x) = (n\cdot f)(x) + (m \cdot f)(x),
    % \end{equation*}
    % e
    % \begin{equation*}
    %     \left(n\cdot (m \cdot f)\right)(x) = n (m \cdot f)(x) = nm f(x) = (nm \cdot f)(x),
    % \end{equation*}
    % portanto \((\mathfrak{F}(X), +, \cdot)\) é um módulo sobre \(\mathbb{Z}\).

    Consideremos para um \(x \in X\) a sua função característica,
    \begin{align*}
        \delta_x : X &\to \set{0,1} \subset \mathbb{Z}\\
                   y &\mapsto \begin{cases}
                         1,&\text{se }y = x\\
                         0,&\text{se }y \neq x,
                     \end{cases}
    \end{align*}
    cujo suporte é \(\set{x}\), portanto \(\delta_x \in \mathfrak{F}(X)\). Mostraremos que \(\mathfrak{F}(X)\) é um módulo livre por admitir como base o conjunto \(B = \setc{\delta_x}{x \in X}\). Seja \(Y \subset X\) um conjunto finito e suponhamos que exista uma aplicação \(a : Y \to \mathbb{Z}\) tal que
    \begin{equation*}
        \sum_{y \in Y} a(y) \delta_y = 0.
    \end{equation*}
    Seja \(A : X \to \mathbb{Z}\) a função definida por aquela combinação linear, então é claro que para \(x \notin Y\), temos \(A(x) = 0\) e para \(x \in Y\) temos \(A(x) = a(x)\). Desse modo, \(a\) deve ser identicamente nula, isto é, a única combinação linear de elementos de \(B\) que resulta no elemento neutro é a trivial, portanto \(B\) é linearmente independente. Seja \(f \in \mathfrak{F}(X)\), então \(\supp(f)\) é um conjunto finito. Assim, temos
    \begin{equation*}
        f = \sum_{y \in \supp(f)} f(y) \delta_y
    \end{equation*}
    pois para \(x \notin \supp(f)\) ambas as expressões se anulam e para \(x \in \supp(f)\) ambas as expressões resultam em \(f(x)\). Acabamos de mostrar que \(B\) é um conjunto gerador para \(\mathfrak{F}(X)\), logo \(B\) é uma base para o módulo, concluindo a demonstração.
\end{proof}

\begin{definition}{Grupo abeliano livremente gerado por um conjunto módulo relações}{grupo_livremente_gerado_relações}
    Seja \(X\) um conjunto não vazio e \(\Lambda\) um conjunto arbitrário de índices. Seja \(\family{X_{\lambda}}{\lambda \in \Lambda}\) uma família de conjuntos finitos de \(X\), \(\family{a_{\lambda}}{\lambda \in \Lambda}\) uma família de funções \(a_{\lambda} : X_{\lambda} \to \mathbb{Z}\) e \(R = \family{r_{\lambda}}{\lambda \in \Lambda}\) uma família de elementos de \(\mathfrak{F}(X)\) dados por
    \begin{equation*}
        r_{\lambda} = \sum_{x \in X_{\lambda}} a_{\lambda}(x) \delta_{x},
    \end{equation*}
    chamados de \emph{relações}. O \emph{submódulo de \(\mathfrak{F}(X)\) gerado pelas relações de \(R\)} é o submódulo \(\mathfrak{R}(R) = \lspan_{\mathbb{Z}}\left(R\right)\) e dizemos que \(\mathfrak{F}(X)/\mathfrak{R}(R)\) é o \emph{grupo livremente gerado por \(X\) módulo as relações de \(R\)}.
\end{definition}
\begin{remark}
    Como \(\mathfrak{R}(R)\) é um submódulo, é claro que o quociente \(\mathfrak{F}(R)/\mathfrak{R}(R)\) está definido.
\end{remark}

\begin{lemma}{Conjunto gerador de \(\mathfrak{F}(X)/\mathfrak{R}(R)\)}{conjunto_gerador}
    Seja \(\mathfrak{F}(X)/\mathfrak{R}(R)\) o grupo livremente  gerado por \(X\) módulo as relações de \(R\). Então
    \begin{equation*}
        S = \setc*{[\delta_{x}]_{\mathfrak{R}(R)}}{x \in X}
    \end{equation*}
    é um conjunto gerador de \(\mathfrak{F}(X)/\mathfrak{R}(R)\), interpretando-o como um módulo sobre \(\mathbb{Z}\).
\end{lemma}
\begin{proof}
    Como \(S \subset \mathfrak{F}(X)/\mathfrak{R}(R)\), é claro que \(\lspan_{\mathbb{Z}}S \subset \mathfrak{F}(X)/\mathfrak{R}(R)\). Seja \([f]_{\mathfrak{R}(R)} \in \mathfrak{F}(X)/\mathfrak{R}(R),\) então como \(f\) tem suporte finito e é dado univocamente por
    \begin{equation*}
        f = \sum_{y \in \supp(f)}{f(y) \delta_y},
    \end{equation*}
    segue que
    \begin{equation*}
        [f]_{\mathfrak{R}(R)} = \left[\sum_{y \in \supp(f)}{f(y) \delta_y}\right]_{\mathfrak{R}(R)} = \sum_{y \in \supp(f)}{f(y)\left[\delta_{y}\right]_{\mathfrak{R}(R)}} \in \lspan_{\mathbb{Z}}S,
    \end{equation*}
    portanto \(\lspan_{\mathbb{Z}}S = \mathfrak{F}(X)/\mathfrak{R}(R)\).
\end{proof}

\subsection{Produto tensorial de dois espaços lineares}

\begin{proposition}{Produto tensorial de dois grupos abelianos}{produto_tensorial_grupos}
    Sejam \(A, B\) grupos abelianos, cujos produtos serão denotados por \(+\) e seus elementos neutros por \(0\). O \emph{produto tensorial \(A \otimes B\) de grupos abelianos \(A\) e \(B\)} é definido pelo grupo abeliano \(\mathfrak{F}(A \times B)/\mathfrak{R}(R)\) livremente gerado por \(A \times B\) módulo as relações de \(R = R_A \cup R_B\), onde
    \begin{equation*}
        R_A = \setc*{\delta_{(a + \tilde{a}, b)} - \delta_{(a,b)} - \delta_{(\tilde{a},b)}}{a, \tilde{a} \in A, b \in B}
    \end{equation*}
    e
    \begin{equation*}
        R_B = \setc*{\delta_{(a,b + \tilde{b})} - \delta_{(a,b)} - \delta_{(a,\tilde{b})}}{a \in A, b, \tilde{b} \in B}.
    \end{equation*}
    Denotemos o elemento \([\delta_{(a,b)}]_{\mathfrak{R}(R)} \in A \otimes B\) por \(a \otimes b\), com \(a\in A\) e \(b \in B\). Assim, \(A \otimes B\) é um grupo abeliano que satisfaz
    \begin{equation*}
        a \otimes b + \tilde{a} \otimes b = (a + \tilde{a}) \otimes b\quad\text{e}\quad
        a \otimes b + a \otimes \tilde{b} = a \otimes (b + \tilde{b}),
    \end{equation*}
    seu elemento neutro é \(0 \otimes 0 = a \otimes 0 = 0 \otimes b\), o elemento inverso de \(a \otimes b\) é \((-a) \otimes b = a \otimes (-b)\), para todos \(a,\tilde{a} \in A\) e \(b,\tilde{b} \in B\). Se \(T \in A \otimes B\), então existe \(N \in \mathbb{N}\) tal que \(T\) é a combinação linear por inteiros de \(N\) elementos da forma \(a_i \otimes b_i\), com \(a_i \in A\), \(b_i \in B\), \(i \in \set{1, 2,\dots, N}\),
    \begin{equation*}
        T = \sum_{i = 1}^N a_i \otimes b_i
    \end{equation*}
    e seu elemento inverso é
    \begin{equation*}
        -T = \sum_{i = 1}^N (-a_i) \otimes b_i = \sum_{i = 1}^N a_i \otimes (-b_i).
    \end{equation*}
\end{proposition}
\begin{proof}
    Como \(A\otimes B\) é o quociente do grupo abeliano \(\mathfrak{F}(A \times B)\) pelo subgrupo normal \(\mathfrak{R}(R)\), então \(A \otimes B\) é abeliano. Sejam \(a, \tilde{a} \in A\) e \(b, \tilde{b} \in B\), então por construção temos
    \begin{equation*}
        \delta_{(a + \tilde{a}, b)} - \delta_{(a,b)} - \delta_{(\tilde{a}, b)} \in \mathfrak{R}(R) \iff \delta_{(a + \tilde{a}, b)} \in [\delta_{(a,b)} + \delta_{(\tilde{a},b)}]_{\mathfrak{R}(R)} \iff (a + \tilde{a})\otimes b = a\otimes b + \tilde{a} \otimes b
    \end{equation*}
    e
    \begin{equation*}
        \delta_{(a, b + \tilde{b})} - \delta_{(a,b)} - \delta_{(a, \tilde{b})} \in \mathfrak{R}(R) \iff \delta_{(a, b + \tilde{b})} \in [\delta_{(a,b)} + \delta_{(a, \tilde{b})}]_{\mathfrak{R}(R)} \iff a\otimes (b+\tilde{b}) = a\otimes b + a \otimes \tilde{b}.
    \end{equation*}
    Notemos que para todo \(a \in A\) temos
    \begin{equation*}
        \delta_{(a, 0)} = \delta_{(a, 0)} + \delta_{(a, b)} - \delta_{(a, b + 0)} = -\left(\delta_{(a,b + 0)} - \delta_{(a,0)} - \delta_{(a,b)}\right) \in \mathfrak{R}(R)
    \end{equation*}
    qualquer que seja \(b \in B\), isto é, \(\delta_{(a,0)}\) pertence ao elemento neutro deste grupo quociente. De forma análoga, temos que \(\delta_{(0,b)} \in \mathfrak{R}(R)\), portanto temos \(0 \otimes b = a \otimes 0\) e em particular, \(0 \otimes 0 = 0 \otimes b = a \otimes 0\). Notemos que
    \begin{equation*}
        a \otimes b + 0 \otimes 0 = a \otimes b + 0 \otimes b = a \otimes b.
    \end{equation*}
    Seja \(a \otimes b \in A \otimes B\), então
    \begin{equation*}
        a \otimes b + (-a) \otimes b = (a - a) \otimes b = 0 \otimes b = 0\otimes 0
    \end{equation*}
    e
    \begin{equation*}
        a \otimes b + a \otimes (-b) = a \otimes (b - b) = a \otimes 0 = 0 \otimes 0,
    \end{equation*}
    portanto tanto \((-a) \otimes b\) quanto \(a \otimes (-b)\) são elementos inversos de \(a \otimes b\), donde segue que devemos identificar \(a \otimes (-b) = (-a)\otimes b\) uma vez que só pode haver um único elemento inverso de um elemento do grupo. Essa identificação pode ser vista de outra forma, considerando que
    \begin{equation*}
        \delta_{(-a, b)} - \delta_{(a, -b)} = \left[\delta_{(-a, b)} + \delta_{(a,b)} - \delta_{(0,b)}\right] - \left[\delta_{(a,b)} + \delta_{(a, -b)} - \delta_{(a,0)}\right] + \delta_{(0,b)} - \delta_{(a, 0)} \in \mathfrak{R}(R),
    \end{equation*}
    isto é, \(\delta_{(-a, b)} \in [\delta_{(a, -b)}]\), portanto \((-a)\otimes b = a\otimes(-b)\).

    Seja \(T \in A \otimes B\), então pelo \cref{lem:conjunto_gerador}, sabemos que existe \(N \in \mathbb{N}\) elementos \((a_i, b_i) \in A\times B\) e inteiros \(t_i \in \mathbb{Z}\) com
    \begin{equation*}
        T = \sum_{i = 1}^N t_i (a_i \otimes b_i),
    \end{equation*}
    onde a multiplicação por inteiros é definida como no \cref{lem:módulo_inteiros}. Notemos que para todos \(a \in A\), \(b \in B\) e \(n \in \mathbb{Z}\), temos
    \begin{equation*}
        n (a \otimes b) = \sum_{i = 1}^n\left(a \otimes b\right) = \left(\sum_{i = 1}^n a\right) \otimes b = (n a)\otimes b
    \end{equation*}
    e \(n (a \otimes b) = a \otimes (n b)\) analogamente. Assim, temos
    \begin{equation*}
        T = \sum_{i = 1}^N (t_i a_i) \otimes b_i = \sum_{i = 1}^N a_i \otimes (t_i b_i)
    \end{equation*}
    então podemos sem perdas de generalidade absorver os números inteiros \(t_i\) nos elementos de \(A \times B\), isto é, podemos escrever
    \begin{equation*}
        T = \sum_{i = 1}^N a_i \otimes b_i,
    \end{equation*}
    mudando se necessário os elementos \(a_i, b_i\) escolhidos previamente. Verificamos que \(0 \otimes 0\) é de fato o elemento neutro, pois temos
    \begin{equation*}
        T + 0 \otimes 0 = 0 \otimes 0 + \sum_{i = 1}^N a_i \otimes b_i = (0 \otimes 0 + a_1\otimes b_1) + \sum_{i=2}^N a_i \otimes b_i = \sum_{i = 1}^N a_i \otimes b_i = T.
    \end{equation*}
    Notemos que
    \begin{equation*}
        T + \sum_{i = 1}^N (-a_i)\otimes b_i = \sum_{i = 1}^N \left[a_i \otimes b_i + (-a_i)\otimes b_i\right] = \sum_{i = 1}^N (a_i - a_i) \otimes b_i = \sum_{i = 1}^N 0 \otimes b_i = N (0 \times 0) = 0\times 0
    \end{equation*}
    e, analogamente,
    \begin{equation*}
        T + \sum_{i = 1}^N a_i\otimes (-b_i) = \sum_{i = 1}^N \left[a_i \otimes b_i + a_i\otimes (-b_i)\right] = \sum_{i = 1}^N a_i \otimes 0 = 0,
    \end{equation*}
    portanto concluímos que
    \begin{equation*}
        -T  = \sum_{i = 1}^N (-a_i)\otimes b_i = \sum_{i = 1}^N a_i \otimes (-b_i),
    \end{equation*}
    pela unicidade do elemento inverso.
\end{proof}

\begin{proposition}{Produto tensorial de dois espaços lineares}{produto_tensorial_lineares}
    Sejam \(V, W\) espaços lineares sobre o corpo \(\mathbb{K}\), cujos vetores nulos são denotados por \(0\). O \emph{produto tensorial \(V \otimes_{\mathbb{K}} W\) de espaços lineares \(V\) e \(W\) sobre \(\mathbb{K}\)} é o espaço linear definido pelo grupo abeliano \(\mathfrak{F}(V \otimes W)/\mathfrak{R}(R)\) livremente gerado por \(V \otimes W\) módulo as relações de \(R\), onde
    \begin{equation*}
        R = \setc*{\delta_{(\alpha v)\otimes w} - \delta_{v \otimes (\alpha w)}}{\alpha \in \mathbb{K}, v \in V, w \in W},
    \end{equation*}
    com o produto por escalares definido por
    \begin{align*}
        \cdot : \mathbb{K} \times V \otimes_{\mathbb{K}} W &\to V \otimes_{\mathbb{K}} W\\
        \left(\alpha,\sum_{i = 1}^Nv_i \otimes_{\mathbb{K}}w_i\right) &\mapsto \sum_{i = 1}^N (\alpha v_i) \otimes_{\mathbb{K}} w_i,
    \end{align*}
    em que denotamos \(v \otimes_{\mathbb{K}}w = [ \delta_{v \otimes w}]_{\mathfrak{R}(R)}\). Se \(T \in V \otimes_{\mathbb{K}} W\), então existe \(N \in \mathbb{N}\) tal que \(T\) é a combinação linear por inteiros de \(N\) elementos da forma \(v_i \otimes_\mathbb{K} w_i\), com \(v_i \in V\), \(w_i \in W\), \(i \in \set{1, 2,\dots, N}\),
    \begin{equation*}
        T = \sum_{i = 1}^N v_i \otimes_{\mathbb{K}} w_i
    \end{equation*}
    seu elemento inverso é
    \begin{equation*}
        -T = \sum_{i = 1}^N (-v_i) \otimes_{\mathbb{K}} w_i = \sum_{i = 1}^N v_i \otimes_{\mathbb{K}} (-w_i),
    \end{equation*}
    e se \(\alpha \in \mathbb{K}\), então
    \begin{equation*}
        \alpha \cdot T = \sum_{i = 1}^N (\alpha v_i) \otimes_{\mathbb{K}} w_i = \sum_{i = 1}^N v_i \otimes_{\mathbb{K}} (\alpha w_i).
    \end{equation*}
    O elemento nulo é \(0 \otimes_{\mathbb{K}} 0 = v \otimes_{\mathbb{K}} 0 = 0 \otimes_{\mathbb{K}} w\), e vale
    \begin{equation*}
        v \otimes_{\mathbb{K}} w + \tilde{v} \otimes_{\mathbb{K}} w = (v + \tilde{v}) \otimes_{\mathbb{K}} w\quad\text\quad
        v \otimes_{\mathbb{K}} w + v \otimes_{\mathbb{K}} \tilde{w} = v \otimes_{\mathbb{K}} (w + \tilde{w})
    \end{equation*}
    para todos \(v, \tilde{v} \in V\) e \(w, \tilde{w} \in W\).
\end{proposition}
\begin{proof}
    Como um quociente de um grupo abeliano, segue que \(V \otimes_{\mathbb{K}} W\) é um grupo abeliano aditivo, como na \cref{prop:grupo_quociente}. Sejam \(v \in V\) e \(w\in W\), então por construção temos
    \begin{equation*}
        \alpha \cdot (v \otimes_{\mathbb{K}} w) = (\alpha \cdot v) \otimes_{\mathbb{K}} w = [\delta_{(\alpha v)\otimes w}]_{\mathfrak{R}(R)} = [\delta_{v \otimes (\alpha w)}]_{\mathfrak{R}(R)} = v \otimes_{\mathbb{K}} (\alpha w)
    \end{equation*}
    para todo \(\alpha \in \mathbb{K}\), já que \(\delta_{(\alpha v) \otimes w} - \delta_{v \otimes (\alpha w)} \in \mathfrak{R}(R)\). Notemos que ainda valem as relações aditivas do produto tensorial, pois
    \begin{equation*}
        v \otimes_\mathbb{K} w + \tilde{v} \otimes_\mathbb{K} w = [\delta_{v \otimes w}]_{\mathfrak{R}(R)}+ [\delta_{\tilde{v} \otimes w}]_{\mathfrak{R}(R)}= [\delta_{v \otimes w} +  \delta_{\tilde{v} \otimes w}]_{\mathfrak{R}(R)} = [\delta_{(v + \tilde{v})\otimes w}]_{\mathfrak{R}(R)} = (v + \tilde{v})\otimes_{\mathbb{K}}w
    \end{equation*}
    e analogamente \(v \otimes_{\mathbb{K}} w + v \otimes_{\mathbb{K}} \tilde{w} = v \otimes_{\mathbb{K}} (w + \tilde{w})\), para todos \(v,\tilde{v} \in V\) e \(w, \tilde{w} \in W\). Como antes, temos
    \begin{equation*}
        0 \otimes_{\mathbb{K}} w = [\delta_{0 \otimes w}]_{\mathfrak{R}(R)} = [\delta_{0 \otimes 0}]_{\mathfrak{R}(R)} = 0 \otimes_{\mathbb{K}} 0 = [\delta_{v \otimes 0}]_{\mathfrak{R}(R)} = v \otimes_{\mathbb{K}} 0
    \end{equation*}
    para todos \(v \in V\) e \(w \in W\).

    Sejam \(T, S \in V \otimes_{\mathbb{K}}W\). Pelo \cref{lem:conjunto_gerador}, sabemos que se \(T \in V \otimes_{\mathbb{K}} W\), então existem \(N \in \mathbb{N}\) elementos \(v_i \in V\) e \(w_i \in W\) tais que
    \begin{equation*}
        T = \sum_{i = 1}^N v_i \otimes_{\mathbb{K}} w_i,
    \end{equation*}
    onde já suprimimos a discussão sobre coeficientes inteiros da combinação linear e os absorvemos na escolha dos elementos de \(V\) e de \(W\). Verifiquemos que \(0 \otimes_\mathbb{K} 0\) é o elemento neutro, pois
    \begin{equation*}
        0 \otimes_{\mathbb{K}} 0 + T = (0 \otimes_{\mathbb{K}} 0 + v_1 \otimes_{\mathbb{K}} w_1) + \sum_{i = 2}^N v_i \otimes_{\mathbb{K}} w_i = \sum_{i = 1}^N v_i \otimes_{\mathbb{K}} w_i = T,
    \end{equation*}
    e assim,
    \begin{equation*}
        T + \sum_{i =1}^N (-v_i) \otimes_{\mathbb{K}} w_i = \sum_{i = 1}^N (v_i - v_i) \otimes_{\mathbb{K}} w_i = \sum_{i=1}^N 0 \otimes_{\mathbb{K}} w_i = 0\otimes_\mathbb{K} 0,
    \end{equation*}
    portanto \(-T = \sum_{i = 1}^N (-v_i)\otimes_{\mathbb{K}} w_i\).
    Notemos que se \(1 \in \mathbb{K}\) é a identidade multiplicativa, então
    \begin{equation*}
        1 \cdot T = \sum_{i = 1}^N (1 v_i)\otimes_{\mathbb{K}} w_i = \sum_{i = 1}^N v_i \otimes _{\mathbb{K}} w_i = T,
    \end{equation*}
    portanto \(1\) é a identidade do produto por escalares.
    % Do mesmo jeito, existem \(M \in \mathbb{N}\) elementos \(a_j \in V\) e \(b_j \in W\) tais que
    % \begin{equation*}
    %     S = \sum_{j = 1}^M a_j \otimes_{\mathbb{K}} b_j.
    % \end{equation*}
    Sejam \(\alpha, \beta \in \mathbb{K}\), então  pela definição do produto por escalares ser finitamente aditivo temos \(\alpha \cdot (T + S) = \alpha T + \alpha S\)
    \begin{equation*}
        \alpha \cdot \left( \beta \cdot T\right) = \alpha \cdot \left( \sum_{i = 1}^N (\beta v_i)\otimes_{\mathbb{K}} w_i \right) = \sum_{i = 1}^N (\alpha \beta v_i) \otimes_{\mathbb{K}} w_i = (\alpha \beta) \cdot T,
    \end{equation*}
    \begin{equation*}
        (\alpha + \beta) \cdot T = \sum_{i = 1}^N ((\alpha + \beta) v_i) \otimes_{\mathbb{K}} w_i = \sum_{i = 1}^N \left((\alpha v_i) \otimes_{\mathbb{K}} w_i + (\beta v_i) \otimes_{\mathbb{K}} w_i\right)
    \end{equation*}
    e
    \begin{equation*}
        \alpha \cdot T = \sum_{i = 1}^N (\alpha v_i) \otimes_{\mathbb{K}} w_i = \sum_{i = 1}^N v_i \otimes_{\mathbb{K}} (\alpha w_i).
    \end{equation*}
    Assim, mostramos que \(V \otimes_{\mathbb{K}} W\) é um espaço linear com as propriedades desejadas.
\end{proof}

\begin{lemma}{Base de um produto tensorial de espaços lineares de dimensão finita}{base_produto_tensorial}
    Sejam \(V, W\) espaços lineares de dimensão finita sobre o corpo \(\mathbb{K}\). Se \(\mathscr{B}_V = \set{\vetor{e}_1, \dots, \vetor{e}_{\dim V}} \subset V\) é uma base de \(V\) e \(\mathscr{B}_W = \set{\vetor{f}_1, \dots, \vetor{f}_{\dim W}} \subset W\) é uma base de \(W\), então
    \begin{equation*}
        \mathscr{B}_{V \otimes_{\mathbb{K}} W} = \setc*{\vetor{e}_i \otimes_{\mathbb{K}} \vetor{f}_j}{i \in \set{1, \dots, \dim V}, j \in \set{1, \dots, \dim W}}
    \end{equation*}
    é uma base para \(V \otimes_{\mathbb{K}} W\), isto é, todo tensor \(T \in V \otimes_{\mathbb{K}} W\) pode ser escrito como
    \begin{equation*}
        T = \sum_{i = 1}^{\dim V} \sum_{j = 1}^{\dim W} T^{ij} \vetor{e}_i\otimes_{\mathbb{K}}\vetor{f}_j
    \end{equation*}
    univocamente, onde \(T^{ij} \in \mathbb{K}\) para \(i \in \set{1, \dots, \dim V}\) e \(j \in \set{1, \dots, \dim W}\).
\end{lemma}
\begin{remark}
    Para não carregar a notação, denotaremos este espaço vetorial apenas por \(V \otimes W\) caso não seja necessário distinguir entre \(V \otimes W\) e \(V \otimes_{\mathbb{K}} W\).
\end{remark}
\begin{proof}
    Seja \(T \in V \otimes W\), então existe uma combinação linear de \(N\) elementos da forma \(v \otimes w\) que resulta em \(T\), e escrevemos
    \begin{equation*}
        T = \sum_{k = 1}^N v_{(k)} \otimes w_{(k)},
    \end{equation*}
    onde \(v_{(k)} \in V\) e \(w_{(k)} \in W\) para \(k \in \set{1, \dots, N}\). Como \(\mathscr{B}_V\) e \(\mathscr{B}_W\) são bases para \(V\) e \(W\), respectivamente, para cada \(k \in \set{1, \dots, N}\) existem \(v_{(k)}^i, w_{(k)}^j \in \mathbb{K}\) com \(i \in \set{1, \dots, \dim V}\) e \(j \in \set{1, \dots, \dim W}\) tais que podemos univocamente escrever \(v_{(k)} = \sum_{i = 1}^{\dim V} v_{(k)}^i \vetor{e}_i\) e \(w_{(k)} = \sum_{j = 1}^{\dim W} w_{(k)}^j \vetor{f}_j.\)
    Com isso, temos
    \begin{align*}
        v_{(k)}\otimes w_{(k)} &= \left(\sum_{i = 1}^{\dim V} v_{(k)}^i\vetor{e}_i\right) \otimes \left(\sum_{i = 1}^{\dim W} w_{(k)}^j \vetor{f}_j\right)\\
                               &= \sum_{i = 1}^{\dim V} v_{(k)}^i \left[\vetor{e}_i \otimes \left(\sum_{j = 1}^{\dim W} w_{(k)}^j \vetor{f}_j\right)\right]\\
                               &= \sum_{i = 1}^{\dim V} \sum_{j = 1}^{\dim W} v_{(k)}^i w_{(k)}^j (\vetor{e}_i \otimes \vetor{f}_j)
    \end{align*}
    e concluímos que
    \begin{equation*}
        T = \sum_{k = 1}^N \sum_{i = 1}^{\dim V} \sum_{j = 1}^{\dim W} v_{(k)}^{i} w_{(k)}^{j} \vetor{e}_i \otimes \vetor{f}_j = \sum_{i = 1}^{\dim V} \sum_{j = 1}^{\dim W} T^{ij} \vetor{e}_i \otimes \vetor{f}_j \in \lspan_{\mathbb{K}}\left(\mathscr{B}_{V \otimes_{\mathbb{K}} W}\right),
    \end{equation*}
    onde definimos as \emph{componentes}
    \begin{equation*}
        T^{ij} = \sum_{k = 1}^N v_{(k)}^i w_{(k)}^j \in \mathbb{K}.
    \end{equation*}
    Assim, mostramos que \(\mathscr{B}_{V \otimes_{\mathbb{K}}W}\) é um conjunto gerador para \(V \otimes W\).\footnote{Não sei provar que é linearmente independente sem entrar na propriedade universal, e nas \href{http://denebola.if.usp.br/~jbarata/Notas_de_aula/arquivos/nc-cap02.pdf}{Notas de aula, página 218} é mostrado somente que é um conjunto gerador, sem deixar claro que é linearmente independente. Este exercício já está longo demais, acredito que foge do escopo.}
\end{proof}
\begin{remark}
    Doravante utilizaremos a convenção de Einstein e escreveremos apenas
    \begin{equation*}
        T = T\indices{^{ij}} \vetor{e}_i \otimes \vetor{f}_j
    \end{equation*}
    subentendendo que há somas nos índices repetidos.
\end{remark}

\subsection{Isomorfismo do produto tensorial de espaços lineares no espaço dual de formas bilineares em espaços lineares de dimensão finita}

\begin{proposition}{Isomorfismo de \(V \otimes W\) no dual de formas bilineares em \(V \oplus W\)}{exercício04b}
    Sejam \(V, W\) espaços lineares de dimensão finita sobre o corpo \(\mathbb{K}\) e seja \(\mathscr{M}(V \oplus W)\) o espaço linear de formas bilineares definidas em \(V \oplus W\). Então o seu espaço dual, \(\mathscr{M}(V \oplus W)'\), é isomórfico a \(V \otimes W\).
\end{proposition}
\begin{proof}
    Sejam \(\mathscr{B}_V, \mathscr{B}_W\) bases de \(V, W\) como no \cref{lem:base_produto_tensorial} e consideramos a aplicação
    \begin{align*}
        \Psi : V \otimes W &\to \mathscr{M}(V \oplus W)'\\
                         T &\mapsto \Psi(T)
    \end{align*}
    tal que\footnote{Podemos definir sem utilizar bases, mas a notação fica mais limpa ao utilizá-las.}
    \begin{equation*}
        \inner*{\Psi(T)}{\omega} = T^{ij}\omega(\vetor{e}_{i} \oplus \vetor{f}_{j})
    \end{equation*}
    para \(\omega \in \mathscr{M}(V \oplus W)\), onde \(T = T^{ij}\vetor{e}_i\otimes \vetor{f}_j\). Sejam \(T, S\in V \otimes W\) e \(\alpha \in \mathbb{K}\), então
    \begin{align*}
        \inner{\Psi(T + \alpha S)}{\omega} &= \inner*{\Psi\left((T^{ij} + \alpha S^{ij})\vetor{e}_i\otimes \vetor{f}_j\right)}{\omega}\\
                                           &= (T^{ij} + \alpha S^{ij})\omega(\vetor{e}_i \oplus \vetor{f}_j)\\
                                           &= T^{ij} \omega(\vetor{e}_i\oplus \vetor{f}_j) + \alpha S^{ij} \omega(\vetor{e}_i \oplus \vetor{f}_j)\\
                                           &= \inner{\Psi(T)}{\omega} + \alpha \inner{\Psi(S)}{\omega}\\
                                           &= \inner{\Psi(T) + \alpha \Psi(S)}{\omega}
    \end{align*}
    para todo \(\omega \in \mathscr{M}(V\oplus W)\), portanto \(\Psi\) é linear. Seja \(T \in \Ker(\Psi)\) e consideremos as formas bilineares
    \begin{align*}
        \vetor{m}^{\tilde{i}\tilde{j}} : V \oplus W &\to \mathbb{K}\\
        v^iw^j \vetor{e}_i \oplus \vetor{f}_j &\mapsto v^{\tilde{i}} w^{\tilde{j}}
    \end{align*}
    com \(\tilde{i} \in I = \set{1, \dots, \dim V}, \tilde{j} \in J = \set{1, \dots, \dim W}\), então
    \begin{equation*}
        \inner{\Psi(T)}{\vetor{m}^{\tilde{i}\tilde{j}}} = T^{ij} \vetor{m}^{\tilde{i}\tilde{j}}(\vetor{e}_i\oplus \vetor{f}_j) = T^{ij} \delta\indices{^{\tilde{i}}_{i}}\delta\indices{^{\tilde{j}}_{j}} = T^{\tilde{i}\tilde{j}},
    \end{equation*}
    portanto concluímos que
    \begin{align*}
        T \in \Ker(\Psi) &\iff \forall \omega \in \mathscr{M}(V \oplus W): \inner{\Psi(T)}{\omega} = 0\\
                         &\implies \forall i \in I, j \in J : \inner{\Psi(T)}{\vetor{m}^{ij}} = 0\\
                         &\iff \forall i \in I, j \in J: T^{ij} = 0\\
                         &\iff T = 0 \otimes 0,
    \end{align*}
    isto é, \(\Ker(\Psi) = \set{0\otimes 0}\), portanto \(\Psi\) é injetora. Como \(\set{\vetor{m}^{ij}}{i \in I, j \in J}\) é uma base para \(\mathscr{M}(V \oplus W)\), temos
    \begin{equation*}
        \omega = \omega(\vetor{e}_i\oplus \vetor{f}_j) \vetor{m}^{ij}
    \end{equation*}
    para todo \(\omega \in \mathscr{M}(V \oplus W)\). Seja \(\Omega \in \mathscr{M}(V \oplus W)'\), então
    \begin{equation*}
        \inner{\Omega}{\omega} = \omega(\vetor{e}_i \oplus \vetor{f}_i) \inner{\Omega}{\vetor{m}^{ij}}
    \end{equation*}
    vale para todo \(\omega\in \mathscr{M}(V \oplus W)\), portanto definindo
    \begin{equation*}
        T = \inner{\Omega}{\vetor{m}^{ij}} \vetor{e}_i \otimes \vetor{f}_j,
    \end{equation*}
    temos
    \begin{equation*}
        \forall \omega \in \mathscr{M}(V\oplus W) : \inner{\Psi(T)}{\omega} = \inner{\Omega}{\omega},
    \end{equation*}
    donde segue que \(\Omega = \Psi(T)\), logo \(\Psi\) é sobrejetora. Assim, mostramos que \(\Psi\) é um isomorfismo linear de \(V \otimes W\) em \(\mathscr{M}(V \oplus W)'\), concluindo a demonstração.
\end{proof}
