\begin{definition}{Ação de um grupo em um conjunto não vazio}{ação}
    Seja \(G\) um grupo, cujo elemento neutro é \(e\), e seja \(M\) um conjunto não vazio. Uma ação à esquerda de \(G\) em \(M\) é uma aplicação
    \begin{align*}
        \alpha : G \times M &\to M\\
                      (g,m) &\mapsto \alpha_g(m)
    \end{align*}
    tal que
    \begin{enumerate}[label=(\alph*)]
        \item \(\alpha_e = \id{M}\);
        \item \(\alpha_g \circ \alpha_{\tilde{g}} = \alpha_{g \tilde{g}}\) para todos \(g, \tilde{g} \in G\);
        \item \(\alpha_g : M \to M\) é bijetora para todo \(g \in G\).
    \end{enumerate}
\end{definition}
\begin{lemma}{Hipótese de bijeção para ação é supérflua}{ação}
    Seja \(G\) um grupo, cujo elemento neutro é \(e\), e seja \(M\) um conjunto não vazio. Se a aplicação
    \begin{align*}
        \alpha : G \times M &\to M\\
                      (g,m) &\mapsto \alpha_g(m)
    \end{align*}
    satisfaz \(\alpha_e = \id{M}\) e \(\alpha_g \circ \alpha_{\tilde{g}} = \alpha_{g \tilde{g}}\) para todos \(g, \tilde{g} \in G\), então \(\alpha\) é uma ação à esquerda de \(G\) em \(M\).
\end{lemma}
\begin{proof}
    Seja \(g \in G\), então \(\alpha_{g} \circ \alpha_{g^{-1}} = \alpha_{g g^{-1}} = \alpha_{e} = \id{M}\) e, analogamente, \(\alpha_{g^{-1}} \circ \alpha_{g} = \id{M}\), portanto \(\alpha_{g}\) admite a aplicação inversa \(\alpha_{g^{-1}}\), isto é, \(\alpha_g\) é uma bijeção.
\end{proof}
