\section{Primeiro teorema de isomorfismos}
\begin{definition}{Subgrupo normal e grupo quociente}{subgrupo_normal}
    Seja \(G\) um grupo. Um subgrupo \(N \subset G\) é dito \emph{normal}, denotado por \(N \normal G\), se para todo \(n \in N\) valer que \(g^{-1} n g \in N\). O grupo quociente \(G/N\) é definido pelo conjunto
    \begin{equation*}
        G / N = \setc{[g] \in \mathbb{P}(G)}{g \in G},
    \end{equation*}
    onde a classe de equivalência de \(g \in G\) é definida por \([g] = \setc{h \in G}{\exists n \in N : g^{-1}h = n}\), e o produto é dado por
    \begin{align*}
        \cdot : G/N \times G/N &\to G/N\\
                     ([g],[h]) &\mapsto [g h],
    \end{align*}
    que é bem definido.
\end{definition}
\begin{proposition}{Primeiro teorema de isomorfismos}{exercício04}
    Sejam \(G\) e \(H\) dois grupos e \(\varphi : G \to H\) um homomorfismo. Os grupos \(G/\Ker(\varphi)\) e \(\Ran(\varphi)\) são isomorfos.
\end{proposition}
\begin{proof}
    Consideremos a aplicação
    \begin{align*}
        \Psi : G/\Ker(\varphi) &\to \Ran(\varphi)\\
                           [g] &\mapsto \varphi(g),
    \end{align*}
    que é bem definida pois se \(g' \in [g]\), então existe \(h \in \Ker(\varphi)\) tal que \(g' = gh\), e temos
    \begin{equation*}
        \varphi(g') = \varphi(gh) = \varphi(g)\varphi(h) = \varphi(g)e_H = \varphi(g).
    \end{equation*}
    Sejam \([g_1], [g_2] \in G/\Ker(\varphi)\) tais que \(\Psi([g_1]) = \Psi([g_2])\), então \(\varphi(g_1) = \varphi(g_2)\). Assim, ao multiplicar ambos os lados por \(\varphi(g_2)^{-1} = \varphi(g_2^{-1})\), obtemos \(\varphi(g_2^{-1}g_1) = e_H\). Portanto, existe \(h \in \Ker(\varphi)\) tal que \(g_2^{-1}g_1 = h,\) isto é, \(g_1 \sim g_2\) e concluímos que \([g_1] = [g_2]\), logo \(\Psi\) é injetora. Que \(\Psi\) é sobrejetora é claro pois
    \begin{equation*}
        \gamma \in \Ran(\varphi) \iff \exists g \in G : \varphi(g) = \gamma \iff \exists g \in G : \Psi([g]) = \gamma,
    \end{equation*}
    portanto \(\Psi\) é bijetora. Temos ainda
    \begin{equation*}
        \Psi([g][h]) = \Psi([gh]) = \varphi(gh) = \varphi(g)\varphi(h) = \Psi([g])\Psi([h]),
    \end{equation*}
    para todos \(g, h \in G\), portanto \(\Psi\) é um isomorfismo.
\end{proof}
