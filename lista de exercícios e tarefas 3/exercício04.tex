\section{Primeiro teorema de isomorfismos}
\begin{definition}{Subgrupo normal e grupo quociente}{subgrupo_normal}
    Seja \(G\) um grupo. Um subgrupo \(N \subset G\) é dito \emph{normal}, denotado por \(N \normal G\), se para todo \(n \in N\) valer que \(g^{-1} n g \in N\). O grupo quociente \(G/N\) é definido pelo conjunto
    \begin{equation*}
        G / N = \setc{[g] \in \mathbb{P}(G)}{g \in G},
    \end{equation*}
    onde a classe de equivalência de \(g \in G\) é definida por \([g] = \setc{h \in G}{\exists n \in N : g^{-1}h = n}\), e o produto é dado por
    \begin{align*}
        \cdot : G/N \times G/N &\to G/N\\
                     ([g],[h]) &\mapsto [g h],
    \end{align*}
    que é bem definido.
\end{definition}

\begin{proposition}{Núcleo e imagem de um homomorfismo são grupos}{núcleo_e_imagem}
    Sejam \(G\) e \(H\) dois grupos e \(\varphi : G \to H\) um homomorfismo. Então
    \begin{equation*}
        \Ker(\varphi) = \setc{g \in G}{\varphi(g) = e_H}
    \end{equation*}
    é um subgrupo normal de \(G\) e
    \begin{equation*}
        \Ran(\varphi) = \setc{h \in H}{\exists g \in G : h = \varphi(g)}
    \end{equation*}
    é um subgrupo de \(H\).
\end{proposition}
\begin{proof}
    Como \(\varphi\) é um homomorfismo, sabemos que \(\varphi(e_G) = e_H\), portanto \(e_G \in \Ker(\varphi)\) e \(e_H \in \Ran(\varphi).\) Sejam \(g_1, g_2 \in \Ker(\varphi)\), então
    \begin{equation*}
        \varphi(g_1g_2) = \varphi(g_1)\varphi(g_2) = e_H e_H = e_H,
    \end{equation*}
    logo \(g_1g_2 \in \Ker(\varphi)\). Se \(g \in \Ker(\varphi)\), então
    \begin{equation*}
        \varphi(g^{-1}) = e_G \varphi(g^{-1}) = \varphi(g) \varphi(g^{-1}) = \varphi(g g^{-1}) = \varphi(e_G) = e_H,
    \end{equation*}
    isto é, \(g^{-1} \in \Ker(\varphi)\). Ainda, se \(\tilde{g} \in G\), temos
    \begin{equation*}
        \varphi(\tilde{g}^{-1} g \tilde{g}) = \varphi(\tilde{g}^{-1}) \varphi(g) \varphi(\tilde{g}) = \varphi(\tilde{g}^{-1}) e_H \varphi(\tilde{g}) = \varphi(\tilde{g})^{-1} \varphi(\tilde{g}) = e_H,
    \end{equation*}
    portanto \(\tilde{g}^{-1} g \tilde{g} \in \Ker(\varphi)\), e concluímos que \(\Ker(\varphi)\) é um subgrupo normal de \(G\).

    Sejam \(h_1, h_2 \in \Ran(\varphi)\), então existem \(g_1, g_2 \in G\) tais que \(h_1 = \varphi(g_1)\) e \(h_2 = \varphi(g_2)\), portanto
    \begin{equation*}
        h_1 h_2 = \varphi(g_1)\varphi(g_2) = \varphi(g_1g_2) \in \Ran(\varphi).
    \end{equation*}
    Se \(h \in \Ran(\varphi)\), existe \(g \in G\) tal que \(h = \varphi(g)\), e, portanto
    \begin{equation*}
        h^{-1} = \varphi(g)^{-1} = \varphi(g^{-1}) \in \Ran(\varphi).
    \end{equation*}
    Desse modo, mostramos que \(\Ran(\varphi)\) é subgrupo de \(H\).
\end{proof}
\begin{proposition}{Primeiro teorema de isomorfismos}{exercício04}
    Sejam \(G\) e \(H\) dois grupos e \(\varphi : G \to H\) um homomorfismo. Os grupos \(G/\Ker(\varphi)\) e \(\Ran(\varphi)\) são isomorfos.
\end{proposition}
\begin{proof}
    Consideremos a aplicação
    \begin{align*}
        \Psi : G/\Ker(\varphi) &\to \Ran(\varphi)\\
                           [g] &\mapsto \varphi(g),
    \end{align*}
    que é bem definida pois se \(\tilde{g} \in [g]\), então existe \(n \in \Ker(\varphi)\) tal que \(g^{-1}\tilde{g} = n\), e temos
    \begin{equation*}
        \varphi(\tilde{g}) = \varphi(gn) = \varphi(g)\varphi(n) = \varphi(g)e_H = \varphi(g),
    \end{equation*}
    isto é, a aplicação independe da escolha de representante da classe de equivalência. Se \([g_1], [g_2] \in G/\Ker(\varphi)\) são tais que \(\Psi([g_1]) = \Psi([g_2])\), então \(\varphi(g_1) = \varphi(g_2)\). Assim, ao multiplicar ambos os lados por \(\varphi(g_2)^{-1} = \varphi(g_2^{-1})\), obtemos \(\varphi(g_2^{-1}g_1) = e_H\). Portanto, existe \(n \in \Ker(\varphi)\) tal que \(g_2^{-1}g_1 = n,\) isto é, \(g_1 \in [g_2]\) e concluímos que \([g_1] = [g_2]\), logo \(\Psi\) é injetora. Que \(\Psi\) é sobrejetora é claro pois
    \begin{equation*}
        h \in \Ran(\varphi) \iff \exists g \in G : \varphi(g) = h \iff \exists g \in G : \Psi([g]) = h,
    \end{equation*}
    portanto \(\Psi\) é bijetora. Temos ainda
    \begin{equation*}
        \Psi([g_1][g_2]) = \Psi([g_1g_2]) = \varphi(g_1g_2) = \varphi(g_1)\varphi(g_2) = \Psi([g_1])\Psi([g_2]),
    \end{equation*}
    para todos \(g_1, g_2 \in G\), portanto \(\Psi\) é um isomorfismo.
\end{proof}
