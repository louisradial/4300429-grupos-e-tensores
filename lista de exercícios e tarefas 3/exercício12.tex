\begin{proposition}{Ação no conjunto das funções definidas em um grupo}{exercício12}
    Sejam \(G\) e \(H\) dois grupos e seja o produto semidireto \(G \rtimes_{\alpha} H\) definido por uma ação à esquerda \(\alpha : G \times H \to H\) de \(G\) sobre \(H\) por automorfismos. Seja \(\mathfrak{F}(H)\) o conjunto de funções definidas em \(H\), então
    \begin{align*}
        \mathscr{A} : (G \rtimes_{\alpha} H) \times \mathfrak{F}(H) &\to \mathfrak{F}(H)\\
                                                          ((g,h),f) &\mapsto \mathscr{A}_{(g,h)}f
    \end{align*}
    define uma ação de \(G \rtimes_{\alpha} H\) em \(\mathfrak{F}(H)\), onde
    \begin{equation*}
        \left(\mathscr{A}_{(g,h)}f\right)(\tilde{h}) = f\left(\alpha_{g^{-1}}(h^{-1}\tilde{h})\right)
    \end{equation*}
    para todo \(\tilde{h} \in H\).
\end{proposition}
\begin{proof}

\end{proof}
