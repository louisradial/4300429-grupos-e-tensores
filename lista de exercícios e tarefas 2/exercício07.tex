\section[Grupo SL(2, C)]{Grupo \(\mathrm{SL}(2, \mathbb{C})\)}
\begin{proposition}{Grupo \(\mathrm{SL}(2, \mathbb{C})\) e as matrizes de Pauli}{exercício7}
    Denotando a combinação linear de matrizes de Pauli por \(\vetor{\alpha} \cdot \vetor{\sigma} = \sum_{k = 1}^3 \alpha_k \sigma_k\) para \(\vetor{\alpha} \in \mathbb{C}^3\), seja \(\mathrm{sl}(2,\mathbb{C}) = \setc{\vetor{\alpha}\cdot\vetor{\sigma}}{\vetor{\alpha}\in \mathbb{C}^3}\). Então
    \begin{equation*}
        \exp(\mathrm{sl}(2,\mathbb{C})) = \setc{a\unity + \vetor{\alpha}\cdot\vetor{\sigma}}{a \in \mathbb{C}\setminus\set{-1}, \vetor{\alpha} \in \mathbb{C}^3 : \alpha_1^2 + \alpha_2^2 + \alpha_3^2 = a^2 - 1}\cup\set{-\unity}
    \end{equation*}
    como um subconjunto próprio de \(\mathrm{SL}(2,\mathbb{C})\).
\end{proposition}
\begin{proof}
    Considere a forma bilinear em \(\mathbb{C}\) definida por
    \begin{align*}
        \omega : \mathbb{C}^3 \times \mathbb{C}^3 &\to \mathbb{C}\\
                                   (\alpha,\beta) &\mapsto \sum_{k=1}^3 \alpha_k \beta_k
    \end{align*}
    e seja \(\vetor{\alpha} \in \mathbb{C}^3\). Mostremos por indução que
    \begin{equation*}
        (\vetor{\alpha} \cdot \vetor{\sigma})^{2n} = \omega(\vetor{\alpha},\vetor{\alpha})^n\unity\quad\text{e}{\quad}(\vetor{\alpha} \cdot \vetor{\sigma})^{2n + 1} = \omega(\vetor{\alpha},\vetor{\alpha})^{n}(\vetor{\alpha} \cdot \vetor{\sigma})
    \end{equation*}
    para todo \(n \in \mathbb{N}\). Sabemos que\footnote{Ver \href{https://github.com/louisradial/4302307-fismat-ii/blob/main/lista de exercícios 2/lista2.pdf}{exercício 6}.} \(\sigma_k \sigma_\ell = \delta_{k \ell}\unity + i \sum_{m = 1}^3 \epsilon_{k \ell m} \sigma_m\) e que \([\sigma_k, \sigma_\ell] = 2i \sum_{m = 1}^3 \epsilon_{k \ell m}\sigma_m\) para todos \(k,\ell \in \set{1,2,3}\). Então, para todo \(\vetor{\alpha} \in \mathbb{C}^3\), temos
    \begin{align*}
        (\vetor{\alpha} \cdot \vetor{\sigma})^2 = \sum_{k = 1}^3 \sum_{\ell = 1}^3\alpha_k \alpha_{\ell} \sigma_\ell \sigma_k
        &= \sum_{k=1}^3\sum_{\ell = 1}^3\alpha_k \alpha_\ell \left(\delta_{k \ell}\unity + i \sum_{m=1}^3 \epsilon_{k \ell m} \sigma_m\right)\\
        &= \left(\sum_{k = 1}^3 \alpha_k^2\right)\unity + \frac12\sum_{k = 1}^3 \sum_{\ell = 1}^3 \alpha_k \alpha_\ell [\sigma_k, \sigma_\ell]\\
        &= \omega(\vetor{\alpha}, \vetor{\alpha}) \unity + \frac12 [\vetor{\alpha}\cdot \vetor{\sigma}, \vetor{\alpha}\cdot\vetor{\sigma}]\\
        &= \omega(\vetor{\alpha},\vetor{\alpha})\unity,
    \end{align*}
    portanto
    \begin{equation*}
        (\vetor{\alpha}\cdot\vetor{\sigma})^3 = (\vetor{\alpha}\cdot\vetor{\sigma})^2(\vetor{\alpha}\cdot\vetor{\sigma}) = \omega(\vetor{\alpha}, \vetor{\alpha})\unity (\vetor{\alpha}\cdot \vetor{\sigma}) = \omega(\vetor{\alpha},\vetor{\alpha})(\vetor{\alpha}\cdot\vetor{\sigma}),
    \end{equation*}
    isto é, as expressões são válidas para \(n = 1\). Suponhamos válidas para algum \(m \in \mathbb{N}\), então
    \begin{equation*}
        (\vetor{\alpha}\cdot\vetor{\sigma})^{2m + 2} = (\vetor{\alpha}\cdot \vetor{\sigma})^{2m+1}(\vetor{\alpha}\cdot\vetor{\sigma}) = \omega(\vetor{\alpha},\vetor{\alpha})^{m}(\vetor{\alpha}\cdot \vetor{\sigma})^2 = \omega(\vetor{\alpha}, \vetor{\alpha})^{m+1}\unity
    \end{equation*}
    e
    \begin{equation*}
        (\vetor{\alpha}\cdot\vetor{\sigma})^{2m+3} = (\vetor{\alpha}\cdot\vetor{\sigma})^{2m+2}(\vetor{\alpha}\cdot\vetor{\sigma}) = \omega(\vetor{\alpha},\vetor{\alpha})^{m+1}\unity(\vetor{\alpha}\cdot\vetor{\sigma})= \omega(\vetor{\alpha},\vetor{\alpha})^{m+1} (\vetor{\alpha}\cdot\vetor{\sigma}),
    \end{equation*}
    portanto são válidas para \(m + 1\in \mathbb{N}\). Pelo princípio da indução finita, concluímos que as expressões são válidas para todo \(n \in \mathbb{N}\). Desse modo, temos
    \begin{align*}
        \exp(\vetor{\alpha}\cdot \vetor{\sigma}) = \unity + \sum_{n = 1}^\infty \frac{(\vetor{\alpha}\cdot\vetor{\sigma})^n}{n!}
        &= \unity + (\vetor{\alpha}\cdot\vetor{\sigma}) + \sum_{n = 1}^\infty \frac{(\vetor{\alpha}\cdot\vetor{\sigma})^{2n}}{(2n)!} + \sum_{n = 1}^\infty \frac{(\vetor{\alpha}\cdot\vetor{\sigma})^{2n+1}}{(2n+1)!}\\
        &= \unity + (\vetor{\alpha}\cdot\vetor{\sigma}) + \left[\sum_{n = 1}^\infty \frac{\omega(\vetor{\alpha},\vetor{\alpha})^n}{(2n)!}\right]\unity + \left[\sum_{n = 1}^\infty \frac{\omega(\vetor{\alpha}, \vetor{\alpha})^n}{(2n+1)!}\right](\vetor{\alpha}\cdot\vetor{\sigma}).
    \end{align*}
    Notemos que se \(\omega(\vetor{\alpha}, \vetor{\alpha}) = 0,\) temos \(\exp(\vetor{\alpha} \cdot{\vetor{\sigma}}) = \unity + \vetor{\alpha}\cdot{\vetor{\sigma}}\), caso contrário temos
    \begin{align*}
        \exp(\vetor{\alpha}\cdot\vetor{\sigma})
        &= \left[\sum_{n = 0}^\infty \frac{\omega(\vetor{\alpha},\vetor{\alpha})^n}{(2n)!}\right]\unity + \left[\sum_{n = 0}^\infty \frac{\omega(\vetor{\alpha}, \vetor{\alpha})^n}{(2n+1)!}\right](\vetor{\alpha}\cdot\vetor{\sigma})\\
        &=\left[\sum_{n=0}^\infty \frac{\sqrt{\omega(\vetor{\alpha}, \vetor{\alpha})}^{2n}}{(2n)!}\right]\unity + \left[\frac{1}{\sqrt{\omega(\vetor{\alpha},\vetor{\alpha})}}\sum_{n = 0}^\infty\frac{\sqrt{\omega(\vetor{\alpha},\vetor{\alpha})}^{2n+1}}{(2n+1)!}\right](\vetor{\alpha}\cdot \vetor{\sigma})\\
        &= \cosh\left[\sqrt{\omega(\vetor{\alpha},\vetor{\alpha})}\right] \unity + \frac{\sinh\left[\sqrt{\omega(\vetor{\alpha}, \vetor{\alpha})}\right]}{\sqrt{\omega(\vetor{\alpha},\vetor{\alpha})}}(\vetor{\alpha}\cdot\vetor{\sigma}),
    \end{align*}
    em que a ramificação da raiz quadrada não afeta o resultado, uma vez que para todo \(\xi \in \mathbb{C}\) valem \(\sinh(-\xi) = -\sinh(\xi)\) e \(\cosh(-\xi) = \cosh(\xi)\).
    No caso em que \(\omega(\vetor{\alpha}, \vetor{\alpha}) = 0\), temos
    \begin{equation*}
        \det[\exp(\vetor{\alpha}\cdot\vetor{\sigma})] = \det \begin{pmatrix}
            1 + \alpha_3 & \alpha_1 - i \alpha_2\\
            \alpha_1 + i \alpha_2 & 1 - \alpha_3
        \end{pmatrix} = 1 - \alpha_3^2 - \alpha_1^2 - \alpha_2^2 = 1 - \omega(\vetor{\alpha}, \vetor{\alpha}) = 1
    \end{equation*}
    e no caso em que \(\omega(\vetor{\alpha}, \vetor{\alpha}) \neq 0\) temos
    \begin{align*}
        \det[\exp(\vetor{\alpha}\cdot\vetor{\sigma})]
        &= \det \begin{pmatrix}
            \cosh\left[\sqrt{\omega(\vetor{\alpha},\vetor{\alpha})}\right] + \frac{\alpha_3 \sinh\left[\sqrt{\omega(\vetor{\alpha},\vetor{\alpha})}\right]}{\sqrt{\omega(\vetor{\alpha},\vetor{\alpha})}} & \frac{(\alpha_1 - i \alpha_2) \sinh\left[\sqrt{\omega(\vetor{\alpha},\vetor{\alpha})}\right]}{\sqrt{\omega(\vetor{\alpha},\vetor{\alpha})}}\\
            \frac{(\alpha_1 + i \alpha_2) \sinh\left[\sqrt{\omega(\vetor{\alpha},\vetor{\alpha})}\right]}{\sqrt{\omega(\vetor{\alpha},\vetor{\alpha})}} & \cosh\left[\sqrt{\omega(\vetor{\alpha},\vetor{\alpha})}\right] - \frac{\alpha_3 \sinh\left[\sqrt{\omega(\vetor{\alpha},\vetor{\alpha})}\right]}{\sqrt{\omega(\vetor{\alpha},\vetor{\alpha})}}\\
        \end{pmatrix}\\
        &= \cosh^2\left[\sqrt{\omega(\vetor{\alpha},\vetor{\alpha})}\right] - \frac{\sinh^2\left[\sqrt{\omega(\vetor{\alpha},\vetor{\alpha})}\right]}{\omega(\vetor{\alpha}, \vetor{\alpha})}(\alpha_3^2 + \alpha_1^2 + \alpha_2^2)\\
        &= \cosh^2\left[\sqrt{\omega(\vetor{\alpha},\vetor{\alpha})}\right] - \sinh^2\left[\sqrt{\omega(\vetor{\alpha},\vetor{\alpha})}\right] = 1,
    \end{align*}
    isto é, \(\exp(\vetor{\alpha}\cdot \vetor{\sigma}) \in \mathrm{SL}(2, \mathbb{C})\)\footnote{Usando \(\det(e^X) = e^{\Tr(X)}\), esse resultado segue trivialmente.} para todo \(\vetor{\alpha} \in \mathbb{C}^3\). Isto é, \(\setc{\exp(\vetor{\alpha}\cdot\vetor{\sigma})}{\vetor{\alpha} \in \mathbb{C}^3} \subset \mathrm{SL}(2,\mathbb{C})\).

    Seja \(S = \setc{a\unity + \vetor{\alpha}\cdot\vetor{\sigma}}{a \in \mathbb{C}\setminus\set{-1}, \vetor{\alpha} \in \mathbb{C}^3 : \omega(\vetor{\alpha}, \vetor{\alpha})= a^2 - 1}\cup\set{-\unity}\). Notando que \(\omega(i\pi\vetor{e}_1, i\pi \vetor{e}_1) = -\pi^2\), temos \(\exp(i\pi \sigma_1) = -\unity \in S\). Seja \(A \in S \setminus \set{-\unity}\), então existem únicos \(\vetor{\alpha} \in \mathbb{C}^3\) e \(a \in \mathbb{C}\setminus\set{-1}\) com \(\omega(\vetor{\alpha}, \vetor{\alpha}) = a^2 - 1\) tais que \(A = a \unity + \vetor{\alpha}\cdot{\vetor{\sigma}}\), pela independência linear de \(\set{\unity, \sigma_1, \sigma_2, \sigma_3}\).  Se \(a = 1\), então \(\omega(\vetor{\alpha},\vetor{\alpha}) = 0\) e temos \(\exp(\vetor{\alpha}\cdot\vetor{\sigma}) = \unity + \vetor{\alpha}\cdot\vetor{\sigma} = A\). Se \(a \in \mathbb{C} \setminus\set{-1, 1}\), existe \(z\in \mathbb{C}\setminus\set{0}\) tal que \(\cosh{z} = a\), então \(\omega(\vetor{\alpha}, \vetor{\alpha}) = a^2 - 1 = \sinh^2{z} \neq 0\). Assim, definimos \(\vetor{\eta} = \csch{z} \vetor{\alpha},\) que satisfaz \(\omega(z\vetor{\eta},z\vetor{\eta}) = z^2\csch{z}^2 \omega(\vetor{\alpha}, \vetor{\alpha}) = z^2\) e temos
    \begin{equation*}
        A = (\cosh{z})\unity + (\sinh{z}) \vetor{\eta}\cdot \vetor{\sigma} = (\cosh{z})\unity + \frac{\sinh{z}}{z}(z\vetor{\eta})\cdot \vetor{\sigma} = \exp(z \vetor{\eta}\cdot \vetor{\sigma}),
    \end{equation*}
    concluindo que \(S \subset \exp(\mathrm{sl}(2,\mathbb{C}))\).

    Suponha por absurdo que \(\exp(\mathrm{sl}(2,\mathbb{C})) \cap \left[\mathrm{SL}(2,\mathbb{C})\setminus S\right]\) é não vazio, então existe \(\vetor{\beta} \in \mathbb{C}^3\) tal que \(\exp(\vetor{\beta}\cdot \vetor{\sigma}) = -\unity + \vetor{\alpha}\cdot\vetor{\sigma}\) para algum \(\vetor{\alpha} \in \mathbb{C}^3 \setminus \set{\vetor{0}}\) com \(\omega(\vetor{\alpha}, \vetor{\alpha}) = 0\). Notemos que \(\omega(\vetor{\beta},\vetor{\beta}) \neq 0\), caso contrário teríamos \(-\unity + \vetor{\alpha} \cdot \vetor{\sigma} = \exp(\vetor{\beta}\cdot \vetor{\sigma}) = \unity + \vetor{\beta}\cdot\vetor{\sigma}\), o que não pode acontecer uma vez que \(\set{\unity, \sigma_1, \sigma_2, \sigma_3}\) é base de \(\mathrm{Mat}(2, \mathbb{C})\). Dessa forma temos
    \begin{equation*}
        -\unity + \vetor{\alpha}\cdot\vetor{\sigma} = \exp(\vetor{\beta}\cdot \vetor{\sigma}) = \cosh\left[\sqrt{\omega(\vetor{\beta},\vetor{\beta})}\right] \unity + \frac{\sinh\left[\sqrt{\omega(\vetor{\beta}, \vetor{\beta})}\right]}{\sqrt{\omega(\vetor{\beta},\vetor{\beta})}}(\vetor{\beta}\cdot\vetor{\sigma}).
    \end{equation*}
    Da independência linear segue que \(\cosh\left[\sqrt{\omega(\vetor{\beta}, \vetor{\beta})}\right] = -1\), e, por conseguinte, \(\sinh\left[\sqrt{\omega(\vetor{\beta}, \vetor{\beta})}\right] = 0\). Então mostramos que \(-\unity + \vetor{\alpha} \cdot \vetor{\sigma} = -\unity\), o que só seria válido para \(\vetor{\alpha} = \vetor{0}\). Essa contradição nos mostra que \( \exp(\mathrm{sl}(2,\mathbb{C})) \setminus S = \exp(\mathrm{sl}(2,\mathbb{C})) \cap \left[\mathrm{SL}(2,\mathbb{C})\setminus S\right] =\emptyset\), já que \(\exp(\mathrm{sl}(2,\mathbb{C})) \subset \mathrm{SL}(2,\mathbb{C})\). Assim, como \(S \subset \exp(\mathrm{sl}(2,\mathbb{C}))\), segue que \(S = \exp(\mathrm{sl}(2, \mathbb{C})) \subsetneq \mathrm{SL}(2,\mathbb{C})\).
\end{proof}
