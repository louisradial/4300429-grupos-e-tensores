\section{Grupo de Galilei}
\begin{definition}{Matriz de transformação de Galilei}{galilei}
    Denotemos por
    \begin{equation*}
        \mathscr{G} = \setc{G(r, \vetor{v})}{r \in \mathrm{O}(3) \land \vetor{v} \in \mathbb{R}^3}
    \end{equation*}
    o conjunto de todas as transformações de Galilei, exceto reversões temporais, onde
    \begin{align*}
        G : \mathrm{O}(3) \times \mathbb{R}^3 &\to \mathrm{Mat}(4, \mathbb{R})\\
                       (r,\vetor{v}) &\mapsto \begin{pmatrix}
                           1 & \vetor{0}^\intercal\\
                           -\vetor{v} & r
                       \end{pmatrix},
    \end{align*}
    utilizando a notação de blocos. Os \emph{boosts} de Galilei com velocidade \(\vetor{v}\) são denotados por \(G(\vetor{v}) = G(\unity_3, \vetor{v})\).
\end{definition}

\begin{proposition}{Grupo de Galilei}{exercício8a}
    O conjunto \(\mathscr{G}\) é um subgrupo de \(\mathrm{GL}(4,\mathbb{\mathbb{R}})\) em relação ao produto matricial, valendo
    \begin{enumerate}[label=(\roman*)]
        \item \(G(\unity_3, \vetor{0}) = \unity_4\);
        \item \(G(r_1, \vetor{v}_1) G(r_2, \vetor{v}_2) = G(r_1r_2, \vetor{v}_1 + r_1\vetor{v}_2)\), para todos \(r_1,r_2 \in \mathrm{O}(3)\) e \(\vetor{v}_1, \vetor{v}_2 \in \mathbb{R}^3\); e
        \item \(G(r, \vetor{v})^{-1} = G(r^{-1}, -r^{-1}\vetor{v})\), para todos \(r \in \mathrm{O}(3)\) e \(\vetor{v} \in \mathbb{R}^3\).
    \end{enumerate}
\end{proposition}
\begin{proof}
    Temos
    \begin{equation*}
        G(\unity_3, \vetor{0}) = \begin{pmatrix}
            1 & \vetor{0}^\intercal\\
            \vetor{0} & \unity_3
        \end{pmatrix} = \unity_4,
    \end{equation*}
    portanto \(\unity_4 \in \mathscr{G}\) e (i) é válida. Sejam \(r_1, r_2 \in \mathscr{O}(3)\) e \(\vetor{v}_1, \vetor{v}_2 \in \mathbb{R}^3\), então
    \begin{align*}
        G(r_1, \vetor{v}_1)G(r_2, \vetor{v}_2)
        =
        \begin{pmatrix}
            1 & \vetor{0}^\intercal\\
            -\vetor{v}_1 & r_1
        \end{pmatrix}
        \begin{pmatrix}
            1 & \vetor{0}^\intercal\\
            -\vetor{v}_2 & r_2
        \end{pmatrix}
        &=
        \begin{pmatrix}
            1 - \vetor{0}^\intercal\vetor{v}_2 & 1\vetor{0}^\intercal + \vetor{0}^\intercal r_2\\
            -\vetor{v}_1 - r_1 \vetor{v}_2 & -\vetor{v}_1\vetor{0}^\intercal + r_1 r_2
        \end{pmatrix}\\
        &=
        \begin{pmatrix}
            1 & \vetor{0}^\intercal\\
            -(\vetor{v}_1 + r_1 \vetor{v}_2) & r_1 r_2
        \end{pmatrix}\\
        &= G(r_1r_2,\vetor{v}_1 + r_1 \vetor{v}_2),
    \end{align*}
    isto é, (ii) segue e concluímos que \(\mathscr{G}\) é fechado em relação ao produto matricial. De (i) e de (ii) temos (iii), pois
    \begin{align*}
        G(r, \vetor{v})G(r^{-1}, -r^{-1}\vetor{v})
        &= G(r r^{-1}, \vetor{v} + r(-r^{-1})\vetor{v})&
        G(r^{-1}, -r^{-1}\vetor{v})G(r, \vetor{v})
        &= G(r^{-1} r, -r^{-1}\vetor{v} + r^{-1}\vetor{v})\\
        &= G(\unity_3, \vetor{v} - \vetor{v})&
        &= G(\unity_3, r^{-1}(\vetor{v} - \vetor{v}))\\
        &= G(\unity_3, \vetor{0})&
        &= G(\unity_3, \vetor{0})\\
        &= \unity_4&
        &= \unity_4
    \end{align*}
    para todo \(\vetor{v} \in \mathbb{R}^3\) e \(r \in \mathrm{O}(3)\). Assim, \(\mathscr{G}\subset \mathrm{GL}(4, \mathbb{R})\) e todo elemento de \(\mathscr{G}\) tem seu inverso em \(\mathscr{G}\), portanto \(\mathscr{G}\) é um subgrupo de \(\mathrm{GL}(4, \mathbb{R})\).
\end{proof}

\begin{proposition}{Grupo de \emph{boosts} de Galilei}{exercício8b}
    O conjunto de \emph{boosts} de Galilei, \(G(\mathbb{R}^3)\), é um subgrupo do grupo de Galilei, com
    \begin{enumerate}[label=(\alph*)]
        \item \(G(\vetor{0}) = \unity_4\);
        \item \(G(\vetor{v}_1)G(\vetor{v}_2) = G(\vetor{v}_1 + \vetor{v}_2)\), para todos \(\vetor{v}_1, \vetor{v}_2 \in \mathbb{R}^3\); e
        \item \(G(\vetor{v})^{-1} = G(-\vetor{v})\), para todo \(\vetor{v} \in \mathbb{R}^3\).
    \end{enumerate}
\end{proposition}
\begin{proof}
    Da propriedade (i) da \cref{prop:exercício8a}, segue que \(G(\vetor{0}) = G(\unity_3, \vetor{0}) = \unity_4\), portanto podemos concluir que \(\unity_4 \in G(\mathbb{R}^3)\), isto é, (a) é válida. Da propriedade (ii) temos
    \begin{equation*}
        G(\vetor{v_1})G(\vetor{v}_2) = G(\unity_3, \vetor{v}_1) G(\unity_3, \vetor{v}_2) = G(\unity_3 \unity_3, \vetor{v}_1 + \unity_3 \vetor{v}_2) = G(\unity_3, \vetor{v}_1 + \vetor{v}_2) = G(\vetor{v}_1 + \vetor{v}_2),
    \end{equation*}
    para todos \(\vetor{v}_1, \vetor{v}_2 \in \mathbb{R}^3\), portanto concluímos que (b) e que o conjunto de \emph{boosts} de Galilei é fechado em relação ao produto. Das propriedades (a) e (b), vemos que
    \begin{equation*}
        G(\vetor{v})G(-\vetor{v}) = G(\vetor{v} - \vetor{v}) = G(\vetor{0}) = \unity_4,
    \end{equation*}
    isto é, temos (c) e segue que todo elemento de \(G(\mathbb{R}^3)\) tem seu inverso em \(G(\mathbb{R}^3)\), portanto \(G(\mathbb{R}^3)\) é um subgrupo de \(\mathscr{G}\).
\end{proof}

\begin{proposition}{Geradores de boosts e rotações do grupo de Galilei}{exercício8c}
    Os geradores dos \emph{boosts} de Galilei são dados por
    \begin{equation*}
        \mathcal{M}_1 = \begin{pmatrix}
            0 & 0 & 0 & 0\\
            -1 & 0 & 0 & 0\\
            0 & 0 & 0 & 0\\
            0 & 0 & 0 & 0
        \end{pmatrix},\quad
        \mathcal{M}_2 = \begin{pmatrix}
            0 & 0 & 0 & 0\\
            0 & 0 & 0 & 0\\
            -1 & 0 & 0 & 0\\
            0 & 0 & 0 & 0
        \end{pmatrix},\quad\text{e}\quad
        \mathcal{M}_3 = \begin{pmatrix}
            0 & 0 & 0 & 0\\
            0 & 0 & 0 & 0\\
            0 & 0 & 0 & 0\\
            -1 & 0 & 0 & 0
        \end{pmatrix},
    \end{equation*}
    e os geradores das rotações no grupo de Galilei são dados por
    \begin{equation*}
        \mathcal{J}_1 = \begin{pmatrix}
            0 & 0 & 0 & 0\\
            0 & 0 & 0 & 0\\
            0 & 0 & 0 & -1\\
            0 & 0 & 1 & 0
        \end{pmatrix},\quad
        \mathcal{J}_2 = \begin{pmatrix}
            0 & 0 & 0 & 0\\
            0 & 0 & 0 & 1\\
            0 & 0 & 0 & 0\\
            0 & -1 & 0 & 0
        \end{pmatrix},\quad\text{e}\quad
        \mathcal{J}_3 = \begin{pmatrix}
            0 & 0 & 0 & 0\\
            0 & 0 & -1 & 0\\
            0 & 1 & 0 & 0\\
            0 & 0 & 0 & 0
        \end{pmatrix},
    \end{equation*}
    Esses geradores satisfazem as relações de comutação
    \begin{equation*}
        [\mathcal{J}_k, \mathcal{J}_\ell] = \sum_{m=1}^3 \epsilon_{k \ell m}\mathcal{J}_m,\quad
        [\mathcal{M}_k, \mathcal{M}_\ell] = 0,\quad\text{e}\quad
        [\mathcal{J}_k, \mathcal{M}_\ell] = \sum_{m=1}^3 \epsilon_{k \ell m}\mathcal{M}_m,
    \end{equation*}
    para todos \(k, \ell \in \set{1,2,3}\).
\end{proposition}
\begin{proof}
    A \cref{prop:exercício8b} nos mostra que \(v \mapsto G(v\vetor{e_k})\) parametriza um subgrupo uniparamétrico dos \emph{boosts} de Galilei, para \(k \in \set{1,2,3}\). Deste modo, seus geradores são dados por
    \begin{equation*}
        \mathcal{M}_k = \diff*{G(v \vetor{e}_k)}{v}[v=0] = \diff*{\begin{pmatrix}
                1 & \vetor{0}^\intercal\\
                v\vetor{e}_k & \unity_3
        \end{pmatrix}}{v}[v=0] = \begin{pmatrix}
                0 & \vetor{0}^\intercal\\
                \vetor{e}_k & 0_{3\times 3},
        \end{pmatrix}
    \end{equation*}
    onde \(0_{3\times3}\) é a matriz nula de \(\mathrm{Mat}(3,\mathbb{R})\). Consideremos o subgrupo uniparamétrico de \(\mathrm{SO}(3) \subset \mathrm{O}(3)\) em torno dos eixos \(\vetor{e}_k\).


    Notemos que \(\mathcal{M}_k \mathcal{M}_\ell = 0\) para todos \(k,\ell \in \set{1,2,3}\), pois
    \begin{equation*}
        \mathcal{M}_k\mathcal{M}_\ell =
        \begin{pmatrix}
            0 & \vetor{0}^\intercal\\
            -\vetor{e}_k & 0_{3\times3}
        \end{pmatrix}
        \begin{pmatrix}
            0 & \vetor{0}^\intercal\\
            -\vetor{e}_\ell & 0_{3\times3}
        \end{pmatrix} =
        \begin{pmatrix}
            0 & \vetor{0}^\intercal\\
            \vetor{0} & 0_{3\times3}
        \end{pmatrix},
    \end{equation*}
    portanto as relações de comutação dos geradores de \emph{boosts} é \([\mathcal{M}_k,\mathcal{M}_\ell] = 0.\) Notemos que os geradores de rotações no grupo de Galilei são, em notação de blocos, dados por
    \begin{equation*}
        \mathcal{J}_k = \begin{pmatrix}
            0 & \vetor{0}^\intercal\\
            \vetor{0} & J_k
        \end{pmatrix},
    \end{equation*}
    onde \(J_k\) é um gerador de \(\mathrm{SO}(3)\), portanto
    \begin{equation*}
        \mathcal{J}_k\mathcal{J}_\ell =
        \begin{pmatrix}
            0 & \vetor{0}^\intercal\\
            \vetor{0} & J_k
        \end{pmatrix}
        \begin{pmatrix}
            0 & \vetor{0}^\intercal\\
            \vetor{0} & J_\ell
        \end{pmatrix} =
        \begin{pmatrix}
            0 & \vetor{0}^\intercal\\
            \vetor{0} & J_kJ_\ell
        \end{pmatrix}.
    \end{equation*}
    Segue da \cref{prop:exercício4a}, então, que
    \begin{equation*}
        [\mathcal{J}_k,\mathcal{J}_\ell] = \begin{pmatrix}
            0 & \vetor{0}^\intercal\\
            \vetor{0} & [J_k,J_\ell]
        \end{pmatrix}
        = \begin{pmatrix}
            0 & \vetor{0}^\intercal\\
            \vetor{0} & \sum_{m = 1}^3 \epsilon_{k \ell m} J_m
        \end{pmatrix} =
        \sum_{m = 1}^3 \epsilon_{k \ell m} \mathcal{J}_m.
    \end{equation*}
    Por fim, temos
    \begin{align*}
        [\mathcal{J}_k,\mathcal{M}_\ell] &=
        \begin{pmatrix}
            0 & \vetor{0}^\intercal\\
            \vetor{0} & J_k
        \end{pmatrix}
        \begin{pmatrix}
            0 & \vetor{0}^\intercal\\
            -\vetor{e}_\ell & 0_{3\times3}
        \end{pmatrix}-
        \begin{pmatrix}
            0 & \vetor{0}^\intercal\\
            -\vetor{e}_\ell & 0_{3\times3}
        \end{pmatrix}
        \begin{pmatrix}
            0 & \vetor{0}^\intercal\\
            \vetor{0} & J_k
        \end{pmatrix}\\
        &=
        \begin{pmatrix}
            0 & \vetor{0}^\intercal\\
            -J_k\vetor{e}_\ell & 0_{3\times 3}
        \end{pmatrix}-
        \begin{pmatrix}
            0 & \vetor{0}^\intercal\\
            \vetor{0} & 0_{3\times3}
        \end{pmatrix}\\
        &=
        \begin{pmatrix}
            0 & \vetor{0}^\intercal\\
            -\sum_{m=1}^3 \epsilon_{k\ell m}\vetor{e}_m & 0
        \end{pmatrix}\\
        &=
        \sum_{m=1}^3 \epsilon_{k \ell m} \mathcal{M}_m,
    \end{align*}
    pois \(J_k \vetor{e}_\ell = \vetor{e}_k \times \vetor{e}_\ell = \sum_{m=1}^3 \epsilon_{k \ell m}\vetor{e}_m\).
\end{proof}
