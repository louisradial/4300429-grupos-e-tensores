\section[Geometria e o grupo SO(3)]{Geometria e o grupo \(\mathrm{SO}(3)\)}
\begin{exercício}{Geometria e o grupo \(\mathrm{SO}(3)\)}{exercício5}
    Usando apenas considerações geométricas, deduza que o efeito de \(R(\theta, \vetor{\eta})\) (uma rotação de um ângulo \(\theta\)) no sentido anti-horário em torno de um eixo definido por um vetor unitário \(\vetor{\eta} \in \mathbb{R}^3\)) em um vetor qualquer é, de fato, dado pela fórmula de Rodrigues para o grupo \(\mathrm{SO}(3)\).
\end{exercício}
