\section{Ângulos de Euler}
\begin{lemma}{Uma base ortonormal direita define um único elemento de \(\mathrm{SO}(3)\)}{exercício9a}
    Sejam \(\vetor{f}_1, \vetor{f}_2 \in S^2 = \setc{\vetor{x} \in \mathbb{R}^3}{\norm{\vetor{x}}=1}\). Se \(\inner{\vetor{f}_a}{\vetor{f}_b} = \delta_{ab}\) para todos \(a,b \in \set{1,2}\), então existe um único \(R \in \mathrm{SO}(3)\) tal que \(R\vetor{e}_1 = \vetor{f}_1,\) \(R\vetor{e}_2 = \vetor{f}_2\), e \(R\vetor{e}_3 = \vetor{f}_1\times\vetor{f}_2\).
\end{lemma}
\begin{proof}
    Seja \(\vetor{f}_3 = \vetor{f}_1 \times \vetor{f}_2\), então por hipótese \(\set{\vetor{f}_1, \vetor{f}_2, \vetor{f}_3}\) é uma base ortonormal positivamente orientada de \(\mathbb{R}^3\). Seja \(R \in \mathrm{End}(\mathbb{R}^3)\) o operador linear definido por
    \begin{align*}
        R : \mathbb{R}^3 &\to \mathbb{R}^3\\
               \vetor{x} &\mapsto \sum_{k = 1}^3 \inner{\vetor{e}_k}{\vetor{x}}\vetor{f}_k,
    \end{align*}
    cuja linearidade é garantida pela bilinearidade do produto interno euclidiano. Este operador satisfaz \(R\vetor{e}_i = \vetor{f}_i\), já que
    \begin{equation*}
        R\vetor{e}_i = \sum_{k = 1}^3\inner{\vetor{e}_i}{\vetor{e}_k}\vetor{f}_k = \vetor{f}_i
    \end{equation*}
    para todo \(i \in \set{1,2,3}\), portanto \(\det{R} = 1\), já que as bases são positivamente orientadas. Sejam \(\vetor{u}, \vetor{v} \in \mathbb{R}^3\), então
    \begin{align*}
        \inner{R\vetor{u}}{R\vetor{v}}
        &= \inner*{\sum_{k = 1}^3\inner{\vetor{u}}{\vetor{e}_k}\vetor{f}_k}{\sum_{\ell = 1}^3\inner{\vetor{v}}{\vetor{e}_\ell}\vetor{f}_\ell}\\
        &= \sum_{k = 1}^3 \sum_{\ell = 1}^3 \inner{\vetor{u}}{\vetor{e}_k}\inner{\vetor{v}}{\vetor{e}_\ell}\inner{\vetor{f}_k}{\vetor{f}_\ell}\\
        &= \sum_{k = 1}^3 \sum_{\ell = 1}^3\inner*{\vetor{u}}{\inner{\vetor{v}}{\vetor{e}_\ell}\vetor{e}_k} \delta_{k \ell}\\
        &= \inner*{\vetor{u}}{\sum_{k = 1}^3\inner{\vetor{v}}{\vetor{e}_k}\vetor{e}_k}\\
        &= \inner{\vetor{u}}{\vetor{v}},
    \end{align*}
    isto é, \(R \in \mathrm{SO}(3)\). Seja \(\tilde{R} \in \mathrm{SO}(3)\) que satisfaz \(\tilde{R}\vetor{e}_i = \vetor{f}_i\), então para todo \(\vetor{x} \in \mathbb{R}^3\) temos
    \begin{equation*}
        (R - \tilde{R})\vetor{x} = (R - \tilde{R}) \sum_{k = 1}^3 \inner{\vetor{e}_k}{\vetor{x}}\vetor{e}_k = \sum_{k = 1}^3 \inner{\vetor{e}_k}{\vetor{x}} (\vetor{f}_k - \vetor{f}_k) = \vetor{0}.
    \end{equation*}
    Isto é, \(R - \tilde{R}\) é o operador nulo, donde segue que \(\tilde{R} = R\), provando a unicidade.
\end{proof}

\begin{lemma}{Ângulo para que o resultado de uma rotação seja ortogonal a um eixo}{exercício9b}
    Sejam \(\vetor{\alpha}, \vetor{\beta}, \vetor{\eta} \in S^2 = \setc{\vetor{x} \in \mathbb{R}^3}{\norm{\vetor{x}}= 1}\) vetores unitários. Se \(\inner{\vetor{\eta}}{\vetor{\alpha}} = 0\) então existe \(\vartheta \in \mathbb{R}\) tal que \(\inner{R(\theta, \vetor{\eta})\vetor{\alpha}}{\vetor{\beta}} = 0\).
\end{lemma}
\begin{proof}
    Sejam \(A = \inner{\vetor{\alpha}}{\vetor{\beta}}\) e \(B = \inner{\vetor{\eta}\times \vetor{\alpha}}{\vetor{\beta}}\), então existe \(\vartheta \in \mathbb{R}\) tal que
    \begin{equation*}
        A = \sqrt{A^2 + B^2} \sin\vartheta\quad\text{e}\quad B = -\sqrt{A^2 + B^2}\cos\vartheta,
    \end{equation*}
    pois se \(A = B = 0\) as igualdades são trivialmente satisfeitas para qualquer \(\vartheta\) e caso contrário estas expressões definem um ângulo \(\vartheta \in [0,2\pi)\) módulo \(2\pi\).
    Pela \cref{prop:exercício5}, temos
    \begin{align*}
        \inner{R(\vartheta, \vetor{\eta})\vetor{\alpha}}{\vetor{\beta}}
        &= \inner*{(\cos\vartheta)\vetor{\alpha} + (1 - \cos\vartheta)\inner{\vetor{\eta}}{\vetor{\alpha}}\vetor{\eta} + (\sin\vartheta)\vetor{\eta}\times\vetor{\alpha}}{\vetor{\beta}}\\
        &= (\cos\vartheta)\inner{\vetor{\alpha}}{\vetor{\beta}} + (1 - \cos\vartheta)\inner{\vetor{\eta}}{\vetor{\alpha}}\inner{\vetor{\eta}}{\vetor{\beta}} + (\sin\vartheta)\inner{\vetor{\eta}\times\vetor{\alpha}}{\vetor{\beta}}\\
        &= \sqrt{A^2 + B^2}(\cos\vartheta \sin\vartheta - \sin\vartheta \cos\vartheta) = 0,
    \end{align*}
    como desejado.
\end{proof}

\begin{lemma}{Produto vetorial de vetores obtidos a partir de um automorfismo linear}{exercício9c}
    Se \(M \in \mathrm{GL}(3, \mathbb{R})\), então
    \begin{equation*}
        M\vetor{\alpha}\times M\vetor{\beta} = (\det{M})\left(M^\intercal\right)^{-1}(\vetor{\alpha} \times \vetor{\beta})
    \end{equation*}
    para todos \(\vetor{\alpha},\vetor{\beta} \in \mathbb{R}^3\).
\end{lemma}
\begin{proof}
    Sejam \(M \in \mathrm{GL}(3, \mathbb{R})\), \(\vetor{\alpha}, \vetor{\beta}, \vetor{\xi} \in \mathbb{R}^3\) vetores e consideremos \(\det(M)\inner{\vetor{\alpha}\times\vetor{\beta}}{\vetor{\xi}} = \det{(M)} \det{(\vetor{\alpha},\vetor{\beta, \vetor{\xi}})}\), onde usamos o determinante tanto como uma aplicação \(\det : \mathrm{Mat}(3, \mathbb{R}) \to \mathbb{R}\) quanto como a forma de volume no \(\mathbb{R}^3\). Note que, denotando por \([\vetor{\alpha}, \vetor{\beta}, \vetor{\xi}] \in \mathrm{Mat}(3, \mathbb{R})\) a matriz cujas colunas são dadas pelos elementos destes vetores, temos \(M[\vetor{\alpha}, \vetor{\beta}, \vetor{\xi}] = [M\vetor{\alpha}, M\vetor{\beta}, M\vetor{\xi}]\) pois, sendo \(\vetor{x}\) a primeira coluna deste produto matricial, temos para \(k \in \set{1,2,3}\) que
    \begin{equation*}
        \inner{\vetor{e}_k}{\vetor{x}} = \sum_{j = 1}^3 M_{kj}\alpha_j = \inner*{\vetor{e}_k}{\sum_{i = 1}^3\sum_{j = 1}^3 M_{ij} \alpha_j \vetor{e}_i} = \inner{\vetor{e}_k}{M\vetor{\alpha}},
    \end{equation*}
    e analogamente para as outras duas colunas. Deste modo, temos
    \begin{align*}
        \inner{\det(M)\vetor{\alpha}\times \vetor{\beta}}{\vetor{\xi}}
        &= \det(M)\inner{\vetor{\alpha\times \vetor{\beta}}}{\vetor{\xi}}\\
        &= \det(M) \det(\vetor{\alpha}, \vetor{\beta}, \vetor{\xi})\\
        &= \det(M\vetor{\alpha}, M\vetor{\beta}, M\vetor{\xi})\\
        &= \inner{(M\vetor{\alpha})\times (M\vetor{\beta})}{M\vetor{\xi}}\\
        &= \inner*{M^\intercal\left[(M\vetor{\alpha}) \times (M\vetor{\alpha})\right]}{\vetor{\xi}}
    \end{align*}
    para todos \(\vetor{\alpha}, \vetor{\beta}, \vetor{\xi} \in \mathbb{R}^3\). Com isso, mostramos que
    \begin{equation*}
        \forall \vetor{\xi} \in \mathbb{R}^3, \forall \vetor{\alpha}, \vetor{\beta}\in \mathbb{R}^3 : \inner*{\det(M) \vetor{\alpha}\times \vetor{\beta} - M^\intercal\left[(M\vetor{\alpha})\times (M\vetor{\beta})\right]}{\vetor{\xi}} = 0,
    \end{equation*}
    portanto segue que
    \begin{equation*}
        \forall \vetor{\alpha}, \vetor{\beta}\in \mathbb{R}^3 : \det(M) \vetor{\alpha}\times\vetor{\beta} = M^\intercal \left[(M\vetor{\alpha})\times (M\vetor{\beta})\right]
    \end{equation*}
    pela não degenerescência do produto interno euclidiano. Como \(M\) é um automorfismo linear, temos
    \begin{equation*}
        \forall \vetor{\alpha}, \vetor{\beta}\in \mathbb{R}^3 : (M\vetor{\alpha})\times (M\vetor{\beta}) = \det(M) (M^\intercal)^{-1}(\vetor{\alpha}\times \vetor{\beta})
    \end{equation*}
    e o resultado proposto segue.
\end{proof}
\begin{corollary}
    Se \(R \in \mathrm{SO}(3)\), então
    \begin{equation*}
        R\vetor{\alpha}\times R\vetor{\beta} = R(\vetor{\alpha}\times \vetor{\beta})
    \end{equation*}
    para todos \(\alpha, \beta \in \mathbb{R}^3\).
\end{corollary}
\begin{proof}
    Como \(\det R = 1\) e \(R^\intercal = R^{-1}\), o resultado segue do \cref{lem:exercício9c}.
\end{proof}

\begin{lemma}{Rotação em torno de um eixo obtido a partir de uma rotação}{exercício9d}
    Seja \(\tilde{R} \in \mathrm{SO}(3)\). Então
    \begin{equation*}
        R(\theta, \tilde{R}\vetor{\eta}) = \tilde{R}R(\theta, \vetor{\eta})\tilde{R}^{-1}
    \end{equation*}
    para todo \(\theta \in \mathbb{R}\) e \(\vetor{\eta} \in S^2\).
\end{lemma}
\begin{proof}
    Da \cref{prop:exercício5} e do \cref{lem:exercício9c}, temos para todo \(\vetor{\alpha} \in \mathbb{R}^3\) que
    \begin{align*}
        R(\theta, \tilde{R}\vetor{\eta})\tilde{R} \vetor{\alpha} &= (\cos\theta) \tilde{R}\vetor{\alpha} + (1 - \cos\theta) \inner{\tilde{R} \vetor{\eta}}{\tilde{R}\vetor{\alpha}} \tilde{R}\vetor{\eta} + (\sin\theta)\tilde{R}\vetor{\eta}\times \tilde{R}\vetor{\alpha}\\
                                                 &= (\cos\theta) \tilde{R}\vetor{\alpha} + (1 - \cos\theta) \inner{\vetor{\eta}}{\vetor{\alpha}} \tilde{R}\vetor{\eta} + (\sin\theta)\tilde{R}(\vetor{\eta}\times \vetor{\alpha})\\
                                                 &= \tilde{R}\left[(\cos\theta) \vetor{\alpha} + (1 - \cos\theta) \inner{\vetor{\eta}}{\vetor{\alpha}}\vetor{\eta} + (\sin\theta)\left(\vetor{\eta}\times \vetor{\alpha}\right)\right]\\
                                                 &= \tilde{R} R(\theta, \vetor{\eta})\vetor{\alpha},
    \end{align*}
    e o resultado segue.
\end{proof}

\begin{proposition}{Ângulos de Euler}{exercício9}
    Seja \(R \in \mathrm{SO}(3)\), então existem \(\theta \in [0,\pi),\) \(\psi \in [0,2\pi)\) e \(\varphi \in [0,2\pi)\) tais que
    \begin{equation*}
        R = R(\varphi, \vetor{e}_3)R(\theta, \vetor{e}_1)R(\psi, \vetor{e}_3).
    \end{equation*}
\end{proposition}
\begin{proof}
    Seja \(R \in \mathrm{SO}(3)\) e sejam \(\vetor{f}_i = R\vetor{e}_i\). Como \(\vetor{f}_1\) e \(\vetor{f}_3\) são ortogonais, temos do \cref{lem:exercício9b} que existe \(\varphi \in [0,2\pi)\) tal que \(\vetor{n} = R(-\varphi, \vetor{f}_3)\vetor{f}_1\) é ortogonal a \(\vetor{e}_3\). Neste caso, \(\vetor{n} \in \lspan_{\mathbb{R}}\set{\vetor{e}_1,\vetor{e}_2}\) e então existe \(\psi \in [0,2\pi)\) tal que \(R(\psi, \vetor{e}_3)\vetor{e}_1 = \vetor{n}\). Isto é,
    \begin{equation*}
        R(\psi, \vetor{e}_3)\vetor{e}_1 = \vetor{n} = R(-\varphi, R\vetor{e}_3)R\vetor{e}_1 = R R(-\varphi,\vetor{e}_3)\vetor{e}_1
    \end{equation*}
    pelo \cref{lem:exercício9d}. Notemos que \(\vetor{n}\) é ortogonal a tanto \(\vetor{e}_3\) quanto a \(\vetor{f}_3\), já que
    \begin{equation*}
        \inner{\vetor{e}_3}{\vetor{n}} = \inner{R(\psi,\vetor{e}_3)\vetor{e}_3}{R(\psi,\vetor{e}_3)\vetor{e_1}} = \inner{\vetor{e}_3}{\vetor{e}_1},
    \end{equation*}
    e analogamente mudando \(\psi \to -\varphi\), e \(\vetor{e}_i \to \vetor{f}_i\). Portanto, existe \(\theta \in [0,2\pi)\) tal que \(R(\theta, \vetor{n})\vetor{e}_3 = \vetor{f}_3\). Na verdade, podemos limitar \(\theta \in [0, \pi)\) pois poderíamos ter considerado o vetor \(-\vetor{n}\) pela liberdade \(\varphi\) e de \(\psi\) e repetido os demais argumentos.

    Resumindo, mostramos que existem \(\psi, \varphi \in [0,2\pi)\) e \(\theta \in [0,\pi]\) tais que
    \begin{equation*}
        R(\varphi, \vetor{f}_3) R(\theta, \vetor{n}) R(\psi, \vetor{e}_3)\vetor{e}_1 = R(\varphi, \vetor{f}_3) R(\theta, \vetor{n}) \vetor{n} = R(\varphi, \vetor{f}_3) \vetor{n} = R(\varphi, \vetor{f}_3) R(-\varphi, \vetor{f}_3)\vetor{f}_1 = \vetor{f}_1.
    \end{equation*}
    Temos também que
    \begin{equation*}
        R(\varphi, \vetor{f}_3)R(\theta, \vetor{n})R(\psi, \vetor{e}_3)\vetor{e}_3 = R(\varphi, \vetor{f}_3) R(\theta, \vetor{n}) \vetor{e}_3 = R(\varphi,\vetor{f}_3)\vetor{f}_3 = \vetor{f}_3,
    \end{equation*}
    donde segue que
    \begin{equation*}
        R(\varphi, \vetor{f}_3)R(\theta, \vetor{n}) R(\psi, \vetor{e}_3)\vetor{e}_2 = R(\varphi, \vetor{f}_3)R(\theta, \vetor{n}) R(\psi, \vetor{e}_3)(\vetor{e}_3 \times \vetor{e}_1) = \vetor{f}_3 \times \vetor{f}_1 = \vetor{f}_2
    \end{equation*}
    pelo \cref{lem:exercício9c}. Pelo \cref{lem:exercício9a}, concluímos que \(R = R(\varphi, \vetor{f}_3)R(\theta, \vetor{n}) R(\psi, \vetor{e}_3)\). Pelo \cref{lem:exercício9d}, sabemos que
    \begin{equation*}
        R(\varphi, \vetor{f}_3) = R\left(\varphi, R(\theta, \vetor{n})\vetor{e}_3\right) = R(\theta, \vetor{n})R(\varphi, \vetor{e}_3)R(-\theta, \vetor{n})
    \end{equation*}
    e que
    \begin{equation*}
        R(\theta, \vetor{n}) = R\left(\theta, R(\psi, \vetor{e}_3)\vetor{e}_1\right) = R(\psi, \vetor{e}_3)R(\theta, \vetor{e}_1) R(- \psi, \vetor{e}_3),
    \end{equation*}
    logo
    \begin{align*}
        R = R(\varphi, \vetor{f}_3)R(\theta, \vetor{n})R(\psi, \vetor{e}_3)
        &= R(\theta, \vetor{n})R(\varphi, \vetor{e}_3)R(-\theta, \vetor{n})R(\theta, \vetor{n})R(\psi, \vetor{e}_3) \\
        &=R(\psi, \vetor{e}_3)R(\theta, \vetor{e}_1) R(- \psi, \vetor{e}_3)R(\varphi, \vetor{e}_3) R(\psi, \vetor{e}_3)\\
        &= R(\psi, \vetor{e}_3)R(\theta, \vetor{e}_1) R(\varphi, \vetor{e}_3),
    \end{align*}
    como desejado.
\end{proof}
