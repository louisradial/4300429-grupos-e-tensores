\section{Ângulos de Euler}
\begin{proposition}{Ângulos de Euler}{exercício9}
    Seja \(\set{\vetor{e}_1, \vetor{e}_2, \vetor{e}_3}\subset \mathbb{R}^3\) a base canônica de \(\mathbb{R}^3\) e denotemos a matriz de rotação por um ângulo \(\vartheta \in [0,2\pi]\) no sentido anti-horário em torno do eixo \(\vetor{e}_k\) por \(R_k(\vartheta) \in \mathrm{SO}(3)\). Seja \(R \in \mathrm{SO}(3)\), então existem \(\theta \in (0,\pi),\) \(\varphi \in [0,2\pi)\) e \(\psi \in [0,2\pi)\) tais que \(R = R_3(\varphi)R_1(\theta)R_3(\psi)\).
\end{proposition}

