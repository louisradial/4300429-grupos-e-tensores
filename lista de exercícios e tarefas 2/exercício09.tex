\section{Ângulos de Euler}

\begin{lemma}{Uma base ortonormal de um plano define um único elemento de \(\mathrm{SO}(3)\)}{exercício9a}
    Sejam \(\vetor{f}_1, \vetor{f}_2 \in S^2 = \setc{\vetor{x} \in \mathbb{R}^3}{\norm{\vetor{x}}=1}\). Se \(\inner{\vetor{f}_a}{\vetor{f}_b} = \delta_{ab}\) para todos \(a,b \in \set{1,2}\), então existe um único \(R \in \mathrm{SO}(3)\) tal que \(R\vetor{e}_1 = \vetor{f}_1,\) \(R\vetor{e}_2 = \vetor{f}_2\), e \(R\vetor{e}_3 = \vetor{f}_1\times\vetor{f}_2\).
\end{lemma}
\begin{proof}
    Seja \(\vetor{f}_3 = \vetor{f}_1 \times \vetor{f}_2\), então por hipótese \(\set{\vetor{f}_1, \vetor{f}_2, \vetor{f}_3}\) é uma base ortonormal positivamente orientada de \(\mathbb{R}^3\). Seja \(R \in \mathrm{End}(\mathbb{R}^3)\) o operador linear definido por
    \begin{align*}
        R : \mathbb{R}^3 &\to \mathbb{R}^3\\
               \vetor{x} &\mapsto \sum_{k = 1}^3 \inner{\vetor{x}}{\vetor{e}_k}\vetor{f}_k,
    \end{align*}
    cuja linearidade é garantida pela bilinearidade do produto interno euclidiano. Este operador satisfaz \(R\vetor{e}_i = \vetor{f}_i\), já que
    \begin{equation*}
        R\vetor{e}_i = \sum_{k = 1}^3\inner{\vetor{e}_i}{\vetor{e}_k}\vetor{f}_k = \vetor{f}_i
    \end{equation*}
    para todo \(i \in \set{1,2,3}\), portanto \(\det{R} = 1\), já que as bases são positivamente orientadas. Sejam \(\vetor{u}, \vetor{v} \in \mathbb{R}^3\), então
    \begin{align*}
        \inner{R\vetor{u}}{R\vetor{v}}
        &= \inner*{\sum_{k = 1}^3\inner{\vetor{u}}{\vetor{e}_k}\vetor{f}_k}{\sum_{\ell = 1}^3\inner{\vetor{v}}{\vetor{e}_\ell}\vetor{f}_\ell}\\
        &= \sum_{k = 1}^3 \sum_{\ell = 1}^3 \inner{\vetor{u}}{\vetor{e}_k}\inner{\vetor{v}}{\vetor{e}_\ell}\inner{\vetor{f}_k}{\vetor{f}_\ell}\\
        &= \sum_{k = 1}^3 \sum_{\ell = 1}^3\inner*{\vetor{u}}{\inner{\vetor{v}}{\vetor{e}_\ell}\vetor{e}_k} \delta_{k \ell}\\
        &= \inner*{\vetor{u}}{\sum_{k = 1}^3\inner{\vetor{v}}{\vetor{e}_k}\vetor{e}_k}\\
        &= \inner{\vetor{u}}{\vetor{v}},
    \end{align*}
    isto é, \(R \in \mathrm{SO}(3)\). Seja \(\tilde{R} \in \mathrm{SO}(3)\) que satisfaz \(\tilde{R}\vetor{e}_i = \vetor{f}_i\), então para todo \(\vetor{x} \in \mathbb{R}^3\) temos
    \begin{equation*}
        (R - \tilde{R})\vetor{x} = (R - \tilde{R}) \sum_{k = 1}^3 \inner{\vetor{e}_k}{\vetor{x}}\vetor{e}_k = \sum_{k = 1}^3 \inner{\vetor{e}_k}{\vetor{x}} (\vetor{f}_k - \vetor{f}_k) = \vetor{0}.
    \end{equation*}
    Isto é, \(R - \tilde{R}\) é o operador nulo, donde segue que \(\tilde{R} = R\), provando a unicidade.
\end{proof}

\begin{lemma}{Ângulo para que o resultado de uma rotação seja ortogonal a um eixo}{exercício9b}
    Sejam \(\vetor{\alpha}, \vetor{\beta}, \vetor{\eta} \in S^2 = \setc{\vetor{x} \in \mathbb{R}^3}{\norm{\vetor{x}}= 1}\) vetores unitários. Se \(\inner{\vetor{\eta}}{\vetor{\alpha}} = 0\) então existe \(\theta \in [0,2\pi)\) tal que \(\inner{R(\theta, \vetor{\eta})\vetor{\alpha}}{\vetor{\beta}} = 0\).
\end{lemma}
\begin{proof}
    Como mostramos na \cref{prop:exercício5}, temos
    \begin{equation*}
        R(\vartheta, \vetor{\eta})\vetor{\alpha} = (\cos\vartheta)\vetor{\alpha} + (1 - \cos\vartheta)\inner{\vetor{\eta}}{\vetor{\alpha}}\vetor{\eta} + (\sin\vartheta)\vetor{\eta}\times\vetor{\alpha}
    \end{equation*}
    para todo \(\vartheta \in [0,2\pi)\). Tomando o produto interno com \(\vetor{\beta}\), temos
    \begin{align*}
        \inner{R(\vartheta, \vetor{\eta})\vetor{\alpha}}{\vetor{\beta}}
        &= (\cos\vartheta)\inner{\vetor{\alpha}}{\vetor{\beta}} + (1 - \cos\vartheta)\inner{\vetor{\eta}}{\vetor{\alpha}}\inner{\vetor{\eta}}{\vetor{\beta}} + (\sin\vartheta)\inner{\vetor{\eta}\times\vetor{\alpha}}{\vetor{\beta}}\\
        &= (\cos\vartheta)\inner{\vetor{\alpha}}{\vetor{\beta}} + (\sin\vartheta)\inner{\vetor{\alpha}\times\vetor{\beta}}{\vetor{\eta}}.
    \end{align*}
    Sejam \(A = \inner{\vetor{\alpha}}{\vetor{\beta}}\) e \(B = \inner{\vetor{\alpha}\times \vetor{\beta}}{\vetor{\eta}}\), então existe \(\theta \in [0, 2\pi)\) tal que
    \begin{equation*}
        \inner{R(\vartheta, \vetor{\eta})\vetor{\alpha}}{\vetor{\beta}} = \sqrt{A^2 + B^2}\sin(\theta - \vartheta),
    \end{equation*}
    pois se \(A = B = 0\) essa afirmação vale para todo \(\theta \in \mathbb{R}\), e caso contrário podemos definir \(\theta\) a partir de
    \begin{equation*}
        \sin\theta = \frac{A}{\sqrt{A^2+B^2}}\quad\text{e}\quad\cos\theta =-\frac{B}{\sqrt{A^2 + B^2}},
    \end{equation*}
    e então a afirmação segue. Com isso mostramos que existe \(\theta \in [0,2\pi)\) tal que
    \begin{equation*}
        \inner{R(\theta, \vetor{\eta})\vetor{\alpha}}{\vetor{\beta}} = 0,
    \end{equation*}
    como desejado.
\end{proof}
\begin{proposition}{Ângulos de Euler}{exercício9}
    Seja \(\set{\vetor{e}_1, \vetor{e}_2, \vetor{e}_3}\subset \mathbb{R}^3\) a base canônica de \(\mathbb{R}^3\) e denotemos a matriz de rotação por um ângulo \(\vartheta \in [0,2\pi]\) no sentido anti-horário em torno do eixo \(\vetor{e}_k\) por \(R_k(\vartheta) \in \mathrm{SO}(3)\). Seja \(R \in \mathrm{SO}(3)\), então existem \(\theta \in (0,\pi),\) \(\varphi \in [0,2\pi)\) e \(\psi \in [0,2\pi)\) tais que \(R = R_3(\varphi)R_1(\theta)R_3(\psi)\).
\end{proposition}
\begin{proof}
    Seja \(R \in \mathrm{SO}(3)\), então podemos assumir sem perda de generalidade que \(R\vetor{e}_3 \neq \pm \vetor{e}_3\), pois se assim fosse
\end{proof}
