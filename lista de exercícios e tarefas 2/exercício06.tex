\section[Grupo SO(1,1)]{Grupo \(\mathrm{SO}(1,1)\)}
\begin{definition}{Grupo \(\mathrm{SO}(1,1)\)}{grupo_so11}
    O grupo de invariância da forma bilinear
    \begin{align*}
        \omega : \mathbb{R}^2 \times \mathbb{R}^2 &\to \mathbb{R}\\
                            (\vetor{u},\vetor{v}) &\mapsto \inner{\vetor{u}}{\eta\vetor{v}}_{\mathbb{R}},
    \end{align*}
    onde \(\eta = \begin{smallpmatrix}
        -1 & 0\\
        0 & 1
    \end{smallpmatrix}\) é denotado por \(\mathrm{O}(1,1)\). O subgrupo \(\mathrm{SO}(1,1)\) é dado pelos elementos de \(\mathrm{O}(1,1)\) com determinante unitário.
\end{definition}

\begin{proposition}{Grupo \(\mathrm{SO}(1,1)\)}{exercício6a}
    O grupo \(\mathrm{SO}(1,1)\) é dado por
    \begin{equation*}
        \mathrm{SO}(1,1) = \setc*{a \unity + b \sigma_1}{a, b\in \mathbb{R} \land a^2 - b^2 = 1},
    \end{equation*}
    onde \(\sigma_1 = \begin{smallpmatrix}
        0 & 1\\
        1 & 0
    \end{smallpmatrix}\) é uma matriz de Pauli.
\end{proposition}
\begin{proof}

\end{proof}

\begin{proposition}{Componentes conexas de \(\mathrm{SO}(1,1)\)}{exercício6b}
    Sejam
    \begin{equation*}
        \mathcal{L}_+^\uparrow = \setc*{\sqrt{1 + \xi^2}\unity + \xi \sigma_1}{\xi \in \mathbb{R}}
        \quad\text{e}\quad
        \mathcal{L}_+^\downarrow = \setc*{-\sqrt{1 + \xi^2}\unity + \xi \sigma_1}{\xi \in \mathbb{R}},
    \end{equation*}
    então \(\mathrm{SO}(1,1) = \mathcal{L}_+^\uparrow \cup \mathcal{L}_+^\downarrow\) e \(\mathcal{L}_+^\uparrow \cap \mathcal{L}_+^\downarrow = \emptyset\).
\end{proposition}
