\section{Parametrização de Euler}
\begin{proposition}{Parametrização de Euler}{exercício10}
    Seja \(R \in \mathrm{SO}(3)\) então existem \(\theta \in [0,\pi),\) \(\varphi \in [0,2\pi)\) e \(\psi \in [0,2\pi)\) tais que
    \begin{equation*}
        R = \begin{pmatrix}
            \cos\varphi \cos\psi - \cos\theta\sin\varphi\sin\psi & - \cos\varphi\sin\psi - \cos\theta \sin\varphi\cos\psi & \sin\varphi \sin\theta\\
            \sin\varphi \cos\psi + \cos\theta \cos\varphi \sin\psi &-\sin\varphi\sin\psi + \cos\theta \cos\varphi \cos\psi & - \cos\varphi \sin\theta\\
            \sin\psi \sin\theta & \cos\psi\sin\theta & \cos\theta
        \end{pmatrix}.
    \end{equation*}
\end{proposition}
\begin{proof}
    Pela \cref{prop:exercício9}, existem \(\theta \in [0,\pi)\), \(\varphi \in [0,2\pi)\), e \(\psi \in [0,2\pi)\) tais que
    \begin{equation*}
        R = R_3(\varphi)R_1(\theta)R_3(\psi).
    \end{equation*}
    Como
    \begin{equation*}
        R_1(\vartheta) = \begin{pmatrix}
            1 & 0 & 0\\
            0 & \cos\vartheta & -\sin\vartheta\\
            0 & \sin\vartheta & \cos\vartheta\\
        \end{pmatrix}
        \quad\text{e}\quad
        R_3(\vartheta) = \begin{pmatrix}
            \cos\vartheta & -\sin\vartheta & 0\\
            \sin\vartheta & \cos\vartheta & 0\\
            0 & 0 & 1
        \end{pmatrix}
    \end{equation*}
    são as matrizes de rotação em torno dos eixos \(\vetor{e}_1\) e \(\vetor{e}_3\), respectivamente, temos
    \begin{align*}
        R &=
        R_3(\varphi)
        R_1(\theta)
        R_3(\psi)\\
          &=
        \begin{pmatrix}
            \cos\varphi & -\sin\varphi & 0\\
            \sin\varphi & \cos\varphi & 0\\
            0 & 0 & 1
        \end{pmatrix}
        \begin{pmatrix}
            1 & 0 & 0\\
            0 & \cos\theta & -\sin\theta\\
            0 & \sin\theta & \cos\theta\\
        \end{pmatrix}
        \begin{pmatrix}
            \cos\psi & -\sin\psi & 0\\
            \sin\psi & \cos\psi & 0\\
            0 & 0 & 1
        \end{pmatrix}\\
                  &=
        \begin{pmatrix}
            \cos\varphi & -\sin\varphi & 0\\
            \sin\varphi & \cos\varphi & 0\\
            0 & 0 & 1
        \end{pmatrix}
        \begin{pmatrix}
            \cos\psi & -\sin\psi & 0\\
            \cos\theta \sin\psi & \cos\theta \cos\psi & -\sin\theta\\
            \sin\theta \sin \psi & \sin\theta \cos\psi& \cos\theta
        \end{pmatrix}\\
                  &= \begin{pmatrix}
            \cos\varphi \cos\psi - \cos\theta\sin\varphi\sin\psi & - \cos\varphi\sin\psi - \cos\theta \sin\varphi\cos\psi & \sin\varphi \sin\theta\\
            \sin\varphi \cos\psi + \cos\theta \cos\varphi \sin\psi &-\sin\varphi\sin\psi + \cos\theta \cos\varphi \cos\psi & - \cos\varphi \sin\theta\\
            \sin\psi \sin\theta & \cos\psi\sin\theta & \cos\theta
        \end{pmatrix},
    \end{align*}
    como desejado.
\end{proof}
